\title{Dynamic Programming -- Infinite Horizon}
\author{William M Volckmann II}
\documentclass[12pt]{article}
\usepackage{bm}
\usepackage{amsmath}
\usepackage{amsfonts}
\usepackage{graphicx}
\usepackage{amssymb}
\usepackage{amsthm}
\usepackage{setspace}
\usepackage{amsthm}
\usepackage{mathtools}
\usepackage{enumitem}
\usepackage{xifthen}
\usepackage{titlesec}
\usepackage[normalem]{ulem}
\usepackage[final]{pdfpages}
\usepackage[top=1.25in, left=1.25in, right=1.25in]{geometry}

\newcommand{\C}{\mathbb{C}}
\newcommand{\F}{\mathbb{F}}
\newcommand{\N}{\mathbb{N}}
\newcommand{\Q}{\mathbb{Q}}
\newcommand{\R}{\mathbb{R}}
\newcommand{\Z}{\mathbb{Z}}
\newcommand{\Chi}{\mathcal{X}}
\newcommand{\grad}{\nabla}
\newcommand{\B}{\beta}
\newcommand{\BH}{\hat{\beta}}
\newcommand{\bh}{\hat{\beta}}
\newcommand{\sumn}{\sum_{i=1}^n}
\newcommand{\crit}{c_{\alpha}}
\newcommand{\given}{\; | \;}
\newcommand{\xbar}{\bar{X}_n}
\newcommand{\asim}{\overset{a}{\sim}}
\newcommand{\Lindent}{\hspace{.4cm} \Longrightarrow \hspace{.4cm}}
\renewcommand{\vec}[1]{\mathbf{#1}}
\DeclareMathOperator*{\argmax}{arg\,max}
\DeclareMathOperator*{\argmin}{arg\,min}

\DeclareMathOperator*{\plim}{plim}
\DeclareMathOperator{\rank}{rank}

\newtheorem{theorem}{Theorem}
\theoremstyle{definition}
\newtheorem{definition}{Definition}
\newtheorem{example}{Example}

\setenumerate{itemsep=-1pt, label=\textbf{(\alph*)}}

\begin{document}


\maketitle
\singlespace

\noindent \emph{These notes borrow heavily from Stokey and Lucas, re-written in a way that I find easier to follow. That sometimes means more exposition, more explanation, worked-out examples, and added (occasionally silly) comments. Also probably some added typos and other mistakes. }



\section{Infinite Horizon}
Let's begin by considering the Cobb-Douglas case where $f(k)=k^{\alpha}$ and $u(c)=\ln(c)$. Recall the law of motion of capital from the finite horizon problem,
\begin{equation}
	k_{t+1} = \alpha \beta \frac{ 1 - (\alpha \beta)^{T -t}}{ 1 - (\alpha \beta)^{T -t +1}}k^{\alpha}_{t}. \label{cdlomoc} 
\end{equation}
As $T$ becomes very large, then because $0 < \alpha \beta < 1$, it follows that the coefficient in front of $k_t^{\alpha}$ will become very close to $\alpha \beta$ for most of the sequence. And at this point, we might wonder why we cannot just take the limit as $T$ goes to infinity as the solution for the infinite horizon problem. Doing so would give an solution of
	\[k_{t+1} = \alpha \beta k_t^{\alpha}.	\]

Well, it turns out we \emph{can} just take the limit for the infinite horizon problem, and this is true in general. To prove it, one must first establish that  interchanging $\max$ and $\lim$ operators  is kosher, and this is actually rather difficult to show. 

So nuts to that. Let's try a different approach. The limit suggests that there is a fixed rule that is used in every period. In this example, the rule seems like it should be to take a fixed proportion of your available capital to the next period, in particular, $\alpha \beta$ of it. More generally, we can conjecture that there is a fixed ``savings function'' of the form
	\[	k_{t+1} = g(k_t).	\]
This is fairly intuitive. The social planner problem looks the same from any given period; the only meaningful difference is the capital stock in each particular period. 

But finding that savings function could be challenging. There isn't any general way we can do it from looking at first order and boundary conditions. The change of variable $z_t = k_t / k_{t-1}^{\alpha}$ is specific to the example. So we need a new approach entirely. 



\end{document}
 