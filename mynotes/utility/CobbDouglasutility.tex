\title{Cobb-Douglas Utility Function}
\author{WMV}
\documentclass[12pt]{article}
\usepackage{bm}
\usepackage{amsmath}
\usepackage{amsfonts}
\usepackage{graphicx}
\usepackage{amssymb}
\usepackage{amsthm}
\usepackage{setspace}
\usepackage{amsthm}
\usepackage{mathtools}
\usepackage{enumitem}
\usepackage{xifthen}
\usepackage{titlesec}
\usepackage[normalem]{ulem}
\usepackage[final]{pdfpages}
\usepackage[top=1.25in, left=1.25in, right=1.25in]{geometry}

\newcommand{\C}{\mathbb{C}}
\newcommand{\F}{\mathbb{F}}
\newcommand{\N}{\mathbb{N}}
\newcommand{\Q}{\mathbb{Q}}
\newcommand{\R}{\mathbb{R}}
\newcommand{\Z}{\mathbb{Z}}
\newcommand{\Chi}{\mathcal{X}}
\newcommand{\grad}{\nabla}
\newcommand{\B}{\beta}
\newcommand{\BH}{\hat{\beta}}
\newcommand{\bh}{\hat{\beta}}
\newcommand{\sumn}{\sum_{i=1}^n}
\newcommand{\crit}{c_{\alpha}}
\newcommand{\given}{\; | \;}
\newcommand{\xbar}{\bar{X}_n}
\newcommand{\asim}{\overset{a}{\sim}}
\newcommand{\Lindent}{\hspace{.4cm} \Longrightarrow \hspace{.4cm}}
\renewcommand{\vec}[1]{\mathbf{#1}}
\DeclareMathOperator*{\argmax}{arg\,max}
\DeclareMathOperator*{\argmin}{arg\,min}

\DeclareMathOperator*{\plim}{plim}
\DeclareMathOperator{\rank}{rank}

\newtheorem{theorem}{Theorem}
\theoremstyle{definition}
\newtheorem{definition}{Definition}
\newtheorem{example}{Example}

\setenumerate{itemsep=-1pt, label=\textbf{(\alph*)}}
%\setlength{\jot}{8pt}
%\setlength{\baselineskip}{1.5em}
%\titlespacing*{\section}{0pt}{4ex plus 1ex minus .2ex}{0ex plus .2ex}
%\titlespacing*{\subsection}{0pt}{4ex plus 1ex minus .2ex}{0ex plus .2ex}
%\titlespacing*{\subsubsection}{0pt}{3ex plus 1ex minus .2ex}{0ex plus .2ex}

\newcounter{ProbCounter}
\setcounter{ProbCounter}{1}
\newcommand{\problem}[1][]{%
\ifthenelse{\equal{#1}{}}{\section*{Problem \arabic{ProbCounter}}}
{\section*{Problem \arabic{ProbCounter} (#1)}}%
\stepcounter{ProbCounter}}



\begin{document}


\maketitle
\onehalfspace


We will consider a setting with $L$ commodities. Commodity $\ell$ has price $p_{\ell}$. We consider the general Cobb-Douglas utility function
	\[	u(x)=A \prod_{{\ell}=1}^L x_{\ell}^{\alpha_{\ell}},\]
where $A$ is some (rather meaningless) ``multiplier'' of utility. We assume that all parameters are positive.



\section{Utility Maximization Problem}
The utility maximization problem is 
	\[	\max_{x \geq 0} A \prod_{{\ell}=1}^L x_{\ell}^{\alpha_{\ell}}  \;\; \text{ s.t. } \;\;  p_1x_1 + ... + p_Lx_L \leq w. \]		
The nice thing about utility functions is that they are ordinal. Which is to say, we can take an increasing monotonic transformation of the utility function and find the same maximizers. In this case, taking the logarithm of the utility function will make it easier to work with. So instead, let's solve the transformed utility maximization problem. Furthermore, we can simply remove $A$ from the problem since that also amounts to a monotonic transformation. Thus, the utility \emph{maximizer} problem can be written as
	\[	\max_{x \geq 0} \alpha_1\ln(x_1) + ... + \alpha_L\ln(x_L)  \;\; \text{ s.t. } \;\;  p_1x_1 + ... + p_Lx_L \leq w. \]

While I'm on the topic of monotonic transformations, it's worth noting that we can always take a monotonic transformation of the exponents in such a way that makes them sum to 1. Thus, we will usually just assume that this is already the case, i.e. that $\sum_{\ell=1}^L\alpha_{\ell} =1$. The results are more elegant in this case anyway.	

And one last remark. Any Cobb-Douglas function with $A>0$ and $\alpha_{\ell}>0$ is quasiconcave. That's great and everything, but the logarithmic transformation gives a concave function. Concave functions are often easier to work with. 

    
\subsection{Walrasian Demand and Indirect Utility Function}
	We are essentially solving the utility maximization problem but replaced with $\argmax$. Notice that we cannot have a corner solution in the logarithmic transformation since $\ln(0)$ is undefined. Relating this to the original function, any corner solution would give utility of zero. So if we took any consumption away from the positive commodity and added it to consumption of the zero commodity, utility would improve. So yeah, corner solutions cannot be optimal as long as $p \gg 0$ and $w > 0$. This implicitly relies on the fact that all Cobb-Douglas commodities are ``goods,'' which is shown by the positive marginal utility with respect to any commodity. Indeed, since increasing any $x_{\ell}$ by any amount will increase utility, Cobb-Douglas functions are locally nonsatiated. This means we can assume that Walras' law holds and turn the constraint into an equality. The Lagrangian can then be written as
	\[\mathcal{L}(x, \lambda) =  \alpha_1\ln(x_1) + ... + \alpha_L\ln(x_L) - \lambda(p_1x_1 + ... + p_Lx_L - w).	\]
So doing the usual first order stuff, we have
\begin{align}
	\frac{\partial L(x, \lambda)}{\partial x_{\ell}} = \frac{\alpha_{\ell}}{x_{\ell}} - \lambda p_{\ell} :=0, \label{cdumaxfoc1}\\	
	p_1x_1 + ... + p_Lx_L =w,\\
	x_{\ell} > 0.
\end{align}
From equation \ref{cdumaxfoc1}, we see that $\lambda \neq 0$ because otherwise $\alpha_{\ell}=0$, but we assume that all $\alpha_{\ell} > 0$. Because $\lambda > 0$, then, we can get from equation \ref{cdumaxfoc1} the string of equalities
	\[\frac{p_1}{\alpha_1}x_1 = \hdots= \frac{p_L}{\alpha_L}x_L.	\]
We can write each $x_{\ell}$ in terms of $x_1$ to obtain the system of equations
\begin{align*}
	p_1x_1 &= \frac{p_1}{\alpha_1}\alpha_1x_1,\\
	\vdots \\
	p_Lx_L &= \frac{p_1}{\alpha_1}\alpha_Lx_1.
\end{align*}
The sum of this system is precisely the constraint function, which happens to satisfy Walras' law. So we can write 
	\[\frac{p_1}{\alpha_1}\alpha_1 x_1 + \hdots + \frac{p_1}{\alpha_1}\alpha_Lx_1 = w \Lindent x_1(p,w)=\frac{\alpha_1}{p_1}\frac{w}{\sum_{\ell=1}^L \alpha_{\ell}} = w \frac{\alpha_1}{p_1},\]	
and more generally,
\[	x_k(p,w) = w\frac{\alpha_k}{p_k}.\]


To find the indirect utility function, we can plug the Walrasian demand functions \emph{into the original utility function.} Fair enough:
\begin{align*}
	v(p,w) &=  A \prod_{\ell=1}^L \left(w \frac{\alpha_{\ell}}{p_{\ell}}\right)^{\alpha_{\ell}}.
\end{align*}
	
	

\section{Expenditure Minimization Problem}

The expenditure minimization problem can be written as
	\[\min_{x \geq 0} p_1x_1 + ... + p_Lx_L \;\; \text{ s.t. } \;\; 	A \prod_{{\ell}=1}^L x_{\ell}^{\alpha_{\ell}} \geq u. \]
Notice that this is virtually identical to the cost minimization problem in production theory. So I could simply refer you to those notes for the solution. However, I will do a derivation using the logarithmic form. To that end, let $\hat{u}=\ln(u/A)$. Then we can write expenditure minimization as the solution to
	\[\min_{x \geq 0} p_1x_1 + ... + p_Lx_L \;\; \text{ s.t. } \;\; \alpha_1 \ln(x_1) + ... + \alpha_L \ln(x_L) \geq \hat{u}. \]

	
\subsection{Hicksian Demand and Expenditure Function}

Again, we clearly cannot have any corner solution because the constraint would be undefined. So the Lagrangian is 
	\[\mathcal{L}(x, \lambda) = p_1x_1 + ... + p_Lx_L + \lambda [\hat{u} - \alpha_1 \ln(x_1) - \hdots - \alpha_L \ln(x_L) ],	\]
which gives rise to first order conditions
\begin{align}
	\frac{\partial \mathcal{L}(x, \lambda)}{\partial x_{\ell}} = p_{\ell} - \lambda \frac{\alpha_{\ell}}{x_{\ell}} &= 0, \label{cdempfoc1}\\
	\lambda [\hat{u} - \alpha_1 \ln(x_1) - \hdots - \alpha_L \ln(x_L) ] &=0,\\
	\lambda &\geq 0,\\
	x_{\ell} &> 0.
\end{align}
From equation \ref{cdempfoc1}, we can immediately conclude that $\lambda >0$ because $p_{\ell} > 0$. This means that the constraint is binding. We can also deduce from equation \ref{cdempfoc1} the string of equalities 
	\[\frac{p_1 }{\alpha_1}x_1 = \hdots = \frac{p_L }{\alpha_L}x_L.	\]
Write each $x_{\ell}$ in terms of $x_1$, conforming to what we see in the constraint, to get the system of equations
\begin{align*}
	\alpha_1\ln(x_1) &= \alpha_1\ln\left(\frac{p_1}{\alpha_1}\right) + \alpha_1\ln \left( \frac{\alpha_1}{p_1}\right) + \alpha_1\ln \left(x_1 \right),\\
	\vdots \\
	\alpha_L\ln(x_L) &= \alpha_L\ln\left(\frac{p_1}{\alpha_1}\right) + \alpha_L\ln \left( \frac{\alpha_L}{p_L}\right) + \alpha_L\ln \left(x_1 \right).
\end{align*}
Summing these up gives the constraint, which is also equal to $\hat{u}=\ln(u/A)$. Exponentiating both sides with respect to $e$ and solving for $x_1$ gives
	\[h_1(p,u) = \frac{\alpha_1}{p_1} \left[ \frac{u}{A} \prod_{\ell=1}^L \left(  \frac{p_{\ell}}{\alpha_{\ell}} \right)^{\alpha_{\ell}} \right]^{1/\alpha},	\]
	where $\alpha = \sum_{\ell=1}^L \alpha_{\ell}$. As alluded to earlier, this is the same function as for the conditional factor demand. So the expenditure function, as you should expect at this point, will be the same as the cost function:
	\[e(p,u) = \alpha \left[ \frac{u}{A} \prod_{\ell=1}^L \left(  \frac{p_{\ell}}{\alpha_{\ell}} \right)^{\alpha_{\ell}} \right]^{1/\alpha}.	\]	



  \end{document}
 