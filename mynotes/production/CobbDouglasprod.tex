\title{Cobb-Douglas Production Function}
\author{William M Volckmann II}
\documentclass[12pt]{article}
\usepackage{bm}
\usepackage{amsmath}
\usepackage{amsfonts}
\usepackage{graphicx}
\usepackage{amssymb}
\usepackage{amsthm}
\usepackage{setspace}
\usepackage{amsthm}
\usepackage{mathtools}
\usepackage{enumitem}
\usepackage{xifthen}
\usepackage{titlesec}
\usepackage[normalem]{ulem}
\usepackage[final]{pdfpages}
\usepackage[top=1.25in, left=1.25in, right=1.25in]{geometry}

\newcommand{\C}{\mathbb{C}}
\newcommand{\F}{\mathbb{F}}
\newcommand{\N}{\mathbb{N}}
\newcommand{\Q}{\mathbb{Q}}
\newcommand{\R}{\mathbb{R}}
\newcommand{\Z}{\mathbb{Z}}
\newcommand{\Chi}{\mathcal{X}}
\newcommand{\grad}{\nabla}
\newcommand{\B}{\beta}
\newcommand{\BH}{\hat{\beta}}
\newcommand{\bh}{\hat{\beta}}
\newcommand{\sumn}{\sum_{i=1}^n}
\newcommand{\crit}{c_{\alpha}}
\newcommand{\given}{\; | \;}
\newcommand{\xbar}{\bar{X}_n}
\newcommand{\asim}{\overset{a}{\sim}}
\newcommand{\Lindent}{\hspace{.4cm} \Longrightarrow \hspace{.4cm}}
\renewcommand{\vec}[1]{\mathbf{#1}}
\DeclareMathOperator*{\argmax}{arg\,max}
\DeclareMathOperator*{\argmin}{arg\,min}

\DeclareMathOperator*{\plim}{plim}
\DeclareMathOperator{\rank}{rank}

\newtheorem{theorem}{Theorem}
\theoremstyle{definition}
\newtheorem{definition}{Definition}
\newtheorem{example}{Example}

\setenumerate{itemsep=-1pt, label=\textbf{(\alph*)}}



\begin{document}


\maketitle
\onehalfspace



We will consider a setting with $L+1$ commodities. The first $L$ commodities serve as inputs and the $L+1$th commodity is the output. Input commodity $\ell$ has input price $w_{\ell}$, and the output commodity has price $p$. We consider the general Cobb-Douglas production function
	\[	f(z)=A \prod_{{\ell}=1}^L z_{\ell}^{\alpha_{\ell}},\]
where $A$ represents the total-factor productivity, an exogenous measure of the level of ``technology."

The parameters $\alpha_{\ell}$ acquire an important interpretation in the context of production. Whenever $\sum_{\ell = 1}^L \alpha_{\ell} = 1$, this production function has constant returns to scale, meaning that if inputs double then output will double. On the other hand if $\sum_{\ell = 1}^L \alpha_{\ell} < 1$, the production function has decreasing returns to scale; and if $\sum_{\ell = 1}^L \alpha_{\ell} > 1$ the production function has increasing returns to scale.


\section{Factor Demand and Cost Minimization}
The cost minimization problem is
	\[	\min_{z \geq 0} w_1z_1 + ... + w_{L}z_{L}  \;\; \text{ s.t. } \;\; A \prod_{\ell = 1}^{L} z_{\ell}^{\alpha_{\ell}} \geq q. \]		
    
    
\subsection{Conditional Factor Demands}
	Solving for the conditional factor demands will allow us to derive the cost function. To find the conditional factor demands, we want to find the vector of inputs that solves
	\[	\argmin_{z \geq 0} w_1z_1 + ... + w_{L}z_{L} \;\; \text{ s.t. } \;\; A \prod_{\ell = 1}^{L} z_{\ell}^{\alpha_{\ell}} \geq q.\]	
	Notice that we cannot have a corner solution for $q >0$ because if any $z_{\ell}=0$, then production is zero. Therefore, writing out the Lagrangian, we have
	\[	\mathcal{L}(z, \lambda) = 	w_1z_1 + ... + w_{L}z_{L}  + \lambda \left[q - A\prod_{\ell = 1}^{L} z_{\ell}^{\alpha_{\ell}} \right] . \]
	This leads to first order conditions
\begin{align}
	\frac{\partial \mathcal{L}(z, \lambda)}{\partial z_{\ell}} = w_{\ell} - \lambda \frac{\alpha_{\ell}}{z_{\ell}} f(z) &= 0  \;\; \text{ for all } {\ell}=1,...,L, \label{costfoc1}\\
	 \lambda \left[q - A\prod_{\ell = 1}^{L} z_{\ell}^{\alpha_{\ell}} \right] &=0, \label{costfoc2}\\
     \lambda &\geq 0,\\
     z_{\ell} &\geq 0 \;\; \text{ for all } \ell=1, ..., L.     
\end{align}	
Notice from equation (\ref{costfoc1}) that $\lambda=0$ implies $w_{\ell}=0$, which violates the assumption that $w \gg 0$. Thus, $\lambda > 0$. Also from equation (\ref{costfoc1}), we can write
	\[	  \frac{w_{\ell}}{\alpha_{\ell}}z_{\ell}  = \lambda f(z).	\]
In other words, we have the series of equalities
	\[ \frac{w_1}{\alpha_1}z_1 = \frac{w_2}{\alpha_2}z_2 = ... = \frac{w_L }{\alpha_L}z_L.	\]
We can express every $z_{\ell}$ in terms of $z_1$ to get the following system of equations:
\begin{align*}
	z_1 &= z_1,\\
	z_2 &= \frac{w_1}{w_2}\frac{\alpha_2}{\alpha_1} z_1,\\
	&\mathrel{\makebox[\widthof{=}]{\vdots}} \\
	z_L &= \frac{w_1}{w_L}\frac{\alpha_L}{\alpha_1} z_1.
\end{align*}

Because $\lambda >0$, we have from equation (\ref{costfoc2}) that $f(z)=q$, that is,
\begin{align*}
		f(z) &= Az_1^{\alpha_1} \times z_2^{\alpha_2} \times ... \times x_L^{\alpha_L} \\	
			&= 	Az_1^{\alpha_1} \times \left[\frac{w_1}{w_2}\frac{\alpha_2}{\alpha_1} z_1 \right]^{\alpha_2} \times ... \times \left[ \frac{w_1}{w_L}\frac{\alpha_L}{\alpha_1} z_1 \right]^{\alpha_L} \\
			& =A z_1^{\alpha _ 1 + \alpha_2 + ... + \alpha_L} \times \left[\frac{w_1}{w_2}\frac{\alpha_2}{\alpha_1}  \right]^{\alpha_2} \times ... \times \left[ \frac{w_1}{w_L}\frac{\alpha_L}{\alpha_1}  \right]^{\alpha_L} \\
			& = q.
\end{align*}
For clarity, let $\alpha = \sum_{{\ell}=1}^L \alpha_{\ell}$. Solving for $z_1$, we get
\begin{align*}
		z_1 &=\left ( \frac{q}{A} \times \left[\frac{w_1}{w_1}\frac{\alpha_1}{\alpha_1} \right ]^{\alpha_1} \times \left[\frac{w_2}{w_1}\frac{\alpha_1}{\alpha_2}  \right]^{\alpha_2} \times ... \times \left[ \frac{w_L}{w_1}\frac{\alpha_1}{\alpha_L}  \right]^{\alpha_L} \right)^{1 / \alpha} \\
		& = \left[ \frac{\alpha_1}{w_1} \right] \left (  \frac{q}{A} \left[ \frac{w_1}{\alpha_1}\right]^{\alpha_1}\left[ \frac{w_2}{\alpha_2}\right]^{\alpha_2 } \times ... \times \left[ \frac{w_L}{\alpha_L}\right]^{\alpha_L } \right)^{1 / \alpha}.
\end{align*}
This can be generalized to any $z_k$:
\begin{align*}
	z_k(w,q) &=  \frac{\alpha_k}{w_k}  \left (  \frac{q}{A} \left[ \frac{w_1}{\alpha_1}\right]^{\alpha_1}\left[ \frac{w_2}{\alpha_2}\right]^{\alpha_2 } \times ... \times \left[ \frac{w_L}{\alpha_L}\right]^{\alpha_L } \right)^{1 / \alpha} \\
		& = \frac{\alpha_k}{w_k} \left( \frac{q}{A} \prod_{{\ell}=1}^L \left[\frac{w_{\ell}}{\alpha_{\ell}} \right ] ^{\alpha_{\ell}} \right)^{1 / a}.
\end{align*}
And so we have the conditional factor demand functions. 



\subsection{Cost Function}
To solve the cost function, we can plug each conditional factor demand into the objective function
	\[	w_1z_1 + w_2 z_2 + ... + w_Lz_L.\]
This will look more intimidating than it actually is. Let's write it all out to see exactly what's happening.
\begin{align*}
		w_1z_1 &=w_1 \left[ \frac{\alpha_1}{w_1} \right]  \left( \frac{q}{A} \prod_{{\ell}=1}^L \left[\frac{w_{\ell}}{\alpha_{\ell}} \right ] ^{\alpha_{\ell}} \right)^{1 / a}, \\
		w_2z_2 &=w_2 \left[ \frac{\alpha_2}{w_2} \right]  \left( \frac{q}{A} \prod_{{\ell}=1}^L \left[\frac{w_{\ell}}{\alpha_{\ell}} \right ] ^{\alpha_{\ell}} \right)^{1 / a},\\
			&\mathrel{\makebox[\widthof{=}]{\vdots}} \\
		w_Lz_L &=w_L \left[ \frac{\alpha_L}{w_L} \right] \left( \frac{q}{A} \prod_{{\ell}=1}^L \left[\frac{w_{\ell}}{\alpha_{\ell}} \right ] ^{\alpha_{\ell}} \right)^{1 / a}.
\end{align*}
Right away we can cancel out $w_{k}$ from each line $k$. Furthermore there are many common terms in each equation. Indeed, after the aforementioned cancellations, the only terms that won't be common will be each $\alpha_{\ell}$, which will be summed for $\alpha$ after factoring out like-terms. So the cost function is
	\[c(w,q) = \alpha \left( \frac{q}{A} \prod_{{\ell}=1}^L \left[\frac{w_{\ell}}{\alpha_{\ell}} \right ] ^{\alpha_{\ell}} \right)^{1 / a}.	\]
	
	
	

\section{Output Supply Function and Profit Function}


\subsection{Output Supply Function}

Now that we have the cost function, we can find the output supply function by solving
	\[	\argmax_{q \geq 0} pq	 -  \alpha \left( \frac{q}{A} \prod_{{\ell}=1}^L \left[\frac{w_{\ell}}{\alpha_{\ell}} \right ] ^{\alpha_{\ell}} \right)^{1 / a}.	\]
Notice that we must have $\sum_{{\ell}=1}^L \alpha_{\ell} \leq 1$ to guarantee that the objective function is concave in $q$, so we are not considering the case of increasing returns to scale. If satisfied, then all we really have to do is take the derivative with respect to $q$, set it equal to zero, and solve. What we get is 
\[
	p  -   q^{(1 - \alpha )/ \alpha } \left( \frac{1}{A} \prod_{{\ell}=1}^L \left[\frac{w_{\ell}}{\alpha_{\ell}} \right ] ^{\alpha_{\ell}} \right)^{1 / a}:=0  \Lindent   q = \left ( p A \prod_{{\ell}=1}^L \left[\frac{\alpha_{\ell}}{w_{\ell}} \right ] ^{\alpha_{\ell}}  \right)^{1 / (1 - \alpha)}.
\]
Thus, the the profit maximizing level of output is
	\[	q(p,w) =   \left ( p^{\alpha} A \prod_{{\ell}=1}^L \left[\frac{\alpha_{\ell}}{w_{\ell}} \right ] ^{\alpha_{\ell}}  \right)^{1 / (1 - \alpha)}.\]

Note that this analysis is not applicable if $\sum_{{\ell}=1}^L \alpha_{\ell} = 1$ since the exponents for $q(p,w)$ would be undefined. Thus, we are not considering the case of constant returns to scale. Therefore this analysis only holds for decreasing returns to scale, that is, when $\sum_{{\ell}=1}^L \alpha_{\ell} < 1$. 



\subsection{Profit Function} 
	To find the profit function, we can plug the output supply function into the revenue minus costs equation to get
\[
	 p \left ( p^{\alpha} A \prod_{{\ell}=1}^L \left[\frac{\alpha_{\ell}}{w_{\ell}} \right ] ^{\alpha_{\ell}}  \right)^{1 / (1 - \alpha)}	 -  \alpha \left[ \left ( p^{\alpha} A \prod_{{\ell}=1}^L \left[\frac{\alpha_{\ell}}{w_{\ell}} \right ] ^{\alpha_{\ell}}  \right)^{1 / (1 - \alpha)} \frac{1}{A} \prod_{{\ell}=1}^L \left[\frac{w_{\ell}}{\alpha_{\ell}} \right ] ^{\alpha_i} \right]^{1 / a},
     	\]
which, after some messy algebra, can be expressed as
\[
	\pi(p,w)=\left( 1	 -  \alpha  \right) \left (p A \prod_{{\ell}=1}^L \left[\frac{\alpha_{\ell}}{w_{\ell}} \right ] ^{\alpha_{\ell}}  \right)^{1 / (1 - \alpha)}.
\]




\subsection{Input Demand Functions}
To find the input demand function $z_k(p,w)$, apply Hotelling's lemma to the profit function to obtain
	\[z_k(p,w)=-\frac{\partial \pi(p,w)}{\partial w_k} = \frac{\alpha_k}{w_k} \left (p A \prod_{{\ell}=1}^L \left[\frac{\alpha_{\ell}}{w_{\ell}} \right ] ^{\alpha_{\ell}}  \right)^{1 / (1 - \alpha)}.	\]  \end{document}
 