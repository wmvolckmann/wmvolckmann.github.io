\title{Constant Elasticity of Substitution (CES) Production Function}
\author{William M Volckmann II}
\documentclass[12pt]{article}
\usepackage{bm}
\usepackage{amsmath}
\usepackage{amsfonts}
\usepackage{graphicx}
\usepackage{amssymb}
\usepackage{amsthm}
\usepackage{setspace}
\usepackage{amsthm}
\usepackage{mathtools}
\usepackage{enumitem}
\usepackage{xifthen}
\usepackage{titlesec}
\usepackage[normalem]{ulem}
\usepackage[final]{pdfpages}
\usepackage[top=1.25in, left=1.25in, right=1.25in]{geometry}

\newcommand{\C}{\mathbb{C}}
\newcommand{\F}{\mathbb{F}}
\newcommand{\N}{\mathbb{N}}
\newcommand{\Q}{\mathbb{Q}}
\newcommand{\R}{\mathbb{R}}
\newcommand{\Z}{\mathbb{Z}}
\newcommand{\Chi}{\mathcal{X}}
\newcommand{\grad}{\nabla}
\newcommand{\B}{\beta}
\newcommand{\BH}{\hat{\beta}}
\newcommand{\bh}{\hat{\beta}}
\newcommand{\sumn}{\sum_{i=1}^n}
\newcommand{\crit}{c_{\alpha}}
\newcommand{\given}{\; | \;}
\newcommand{\xbar}{\bar{X}_n}
\newcommand{\asim}{\overset{a}{\sim}}
\newcommand{\Lindent}{\hspace{.4cm} \Longrightarrow \hspace{.4cm}}
\renewcommand{\vec}[1]{\mathbf{#1}}
\DeclareMathOperator*{\argmax}{arg\,max}
\DeclareMathOperator*{\argmin}{arg\,min}

\DeclareMathOperator*{\plim}{plim}
\DeclareMathOperator{\rank}{rank}

\newtheorem{theorem}{Theorem}
\theoremstyle{definition}
\newtheorem{definition}{Definition}
\newtheorem{example}{Example}

\setenumerate{itemsep=-1pt, label=\textbf{(\alph*)}}
%\setlength{\jot}{8pt}
%\setlength{\baselineskip}{1.5em}
%\titlespacing*{\section}{0pt}{4ex plus 1ex minus .2ex}{0ex plus .2ex}
%\titlespacing*{\subsection}{0pt}{4ex plus 1ex minus .2ex}{0ex plus .2ex}
%\titlespacing*{\subsubsection}{0pt}{3ex plus 1ex minus .2ex}{0ex plus .2ex}

\newcounter{ProbCounter}
\setcounter{ProbCounter}{1}
\newcommand{\problem}[1][]{%
\ifthenelse{\equal{#1}{}}{\section*{Problem \arabic{ProbCounter}}}
{\section*{Problem \arabic{ProbCounter} (#1)}}%
\stepcounter{ProbCounter}}



\begin{document}


\maketitle
\singlespace
We will consider a setting with $L+1$ commodities. The first $L$ commodities serve as inputs and the $L+1$th commodity is the output. Input commodity $\ell$ has input price $w_{\ell}$, and the output commodity has price $p$. We consider the general constant elasticity of substitution (CES) production function
	\[	f(z)=A \left(\sum_{\ell = 1}^L \alpha_{\ell} z_{\ell}^{\sigma} \right)^{1 / \sigma},\]
where $A$ represents the total-factor productivity, an exogenous measure of the level of ``technology." The parameters $\alpha_{\ell}$ are \emph{share parameters} that capture how strongly input $\ell$ contributes to production. For instance, a low $\alpha_{\ell}$ implies that input $\ell$ doesn't affect output very much. 


\section{Factor Demand and Cost Minimization}
The cost minimization problem is
	\[	\min_{z \geq 0} w_1z_1 + ... + w_{L}z_{L}  \;\; \text{ s.t. } \;\; A \left(\sum_{\ell = 1}^L \alpha_{\ell} z_{\ell}^{\sigma} \right)^{1 / \sigma} \geq q. \]	
I will first find the conditional factor demands and then use those to derive the cost function. 
    
    
\subsection{Conditional Factor Demands}
	Solving for the conditional factor demands will allow us to derive the cost function. To find the conditional factor demands, we want to find the vector of inputs that solves
	\[	\argmin_{z \geq 0} w_1z_1 + ... + w_{L}z_{L}  \;\; \text{ s.t. } \;\; A \left(\sum_{\ell = 1}^L \alpha_{\ell} z_{\ell}^{\sigma} \right)^{1 / \sigma} \geq q. \]	
	Writing out the Lagrangian, and assuming an interior solution, we have
	\[	\mathcal{L}(z, \lambda) = 	w_1z_1 + ... + w_{L}z_{L}  + \lambda \left[q - A \left(\sum_{\ell = 1}^L \alpha_{\ell} z_{\ell}^{\sigma} \right)^{1 / \sigma} \right] . \]
	(This leads to first order conditions
\begin{align}
	\frac{\partial \mathcal{L}(z, \lambda)}{\partial z_{\ell}} = w_{\ell} - \lambda \left[  \alpha_k z_k^{\sigma - 1} A \left(\sum_{\ell = 1}^L \alpha_{\ell} z_{\ell}^{\sigma} \right)^{(1 - \sigma)/\sigma} \right] &= 0  \;\; \text{ for all } {\ell}=1,...,L, \label{costfoc1}\\
	 \lambda \left[q - A \left(\sum_{\ell = 1}^L \alpha_{\ell} z_{\ell}^{\sigma} \right)^{1 / \sigma} \right] &=0, \label{costfoc2}\\
     \lambda &\geq 0,\\
     z_{\ell} &\geq 0 \;\; \text{ for all } \ell=1, ..., L.     
\end{align}	

Notice that from equation (\ref{costfoc1}), we can't have $\lambda=0$ because otherwise $w_k=0$, whereas we assume $w \gg 0$.  Therefore we can write
	\[	\frac{\alpha_k}{w_k}z_k^{\sigma - 1} = \frac{1}{A \lambda[ f(x)]^{1 - \sigma}}.\]
In other words, we have the series of equalities
	\[ \frac{\alpha_1}{w_1}z_1^{\sigma - 1} =\frac{\alpha_2}{w_2}z_2^{\sigma - 1} = ... = \frac{\alpha_L}{w_L}z_L^{\sigma - 1}.	\]
We can express every $\alpha_{\ell}z_{\ell}^{\sigma}$ in terms of $z_1$ to get the following system of equations:
\begin{align*}
		\alpha_{1}z_1^{\sigma} &= \alpha_1 z_1^{\sigma} \left (\frac{\alpha_1}{w_1} \frac{w_1}{\alpha_1} \right)^{\sigma / (\sigma - 1)},\\
	\alpha_{2}z_2^{\sigma} &= 	\alpha_{2} z_1^{\sigma}\left (\frac{\alpha_1}{w_1} \frac{w_2}{\alpha_2} \right)^{\sigma / (\sigma - 1)},\\
	&\mathrel{\makebox[\widthof{=}]{\vdots}} \\
	\alpha_{L}z_L^{\sigma} &= \alpha_Lz_1^{\sigma} \left (\frac{\alpha_1}{w_1} \frac{w_L}{\alpha_L} \right)^{\sigma / (\sigma - 1)}.
\end{align*}
Since $\lambda > 0$, it follows from equation (\ref{costfoc2}) that we can write the objective function as $f(z)=q$. Taking it a step further, we can write
	\[q^{\sigma} = A^{\sigma} \sum_{\ell = 1}^L \alpha_{\ell} z_{\ell}^{\sigma}. \]
From the above system of equations, we can write
\begin{align*}
	q^{\sigma} &= A^{\sigma} \left[ \alpha_1 z_1^{\sigma} \left (\frac{\alpha_1}{w_1} \frac{w_1}{\alpha_1} \right)^{\sigma / (\sigma - 1)}  + \hdots +  \alpha_L z_1^{\sigma} \left (\frac{\alpha_1}{w_1} \frac{w_L}{\alpha_L} \right)^{\sigma / (\sigma - 1)} \right] \\
		&= A^{\sigma}z_1^{\sigma}  \left(\frac{\alpha_1}{w_1}\right)^{\sigma / (\sigma - 1)} \left[ \alpha_1 \left (\frac{w_1}{\alpha_1} \right)^{\sigma / (\sigma - 1)}  + \hdots +  \alpha_L \left ( \frac{w_L}{\alpha_L} \right)^{\sigma / (\sigma - 1)} \right]	\\
		&= A^{\sigma}z_1^{\sigma}  \left(\frac{\alpha_1}{w_1}\right)^{\sigma / (\sigma - 1)} \sum_{\ell=1}^L \alpha_{\ell} \left( \frac{w_{\ell}}{\alpha_{\ell}} \right)^{\sigma / (\sigma - 1)}.
\end{align*}
So solving for $z_1$, we get
	\[ z_1(q,w)  =\frac{ \dfrac{q}{A} \left(\dfrac{w_1}{\alpha_1}\right)^{1 / (\sigma - 1)} }{\left( \sum_{\ell=1}^L \alpha_{\ell} \left( \dfrac{w_{\ell}}{\alpha_{\ell}} \right)^{\sigma / (\sigma - 1)} \right) ^{1 / \sigma}}.
	\]
This, of course, can be generalized for any input, where
	\[ z_k(q,w)  =\frac{\dfrac{q}{A} \left(\dfrac{w_k}{\alpha_k}\right)^{1 / (\sigma - 1)} }{\left( \sum_{\ell=1}^L \alpha_{\ell} \left( \dfrac{w_{\ell}}{\alpha_{\ell}} \right)^{\sigma / (\sigma - 1)} \right) ^{1 / \sigma}}.
	\]
And so we have the conditional factor demand functions. 



\subsection{Cost Function}
To solve the cost function, we can plug each conditional factor demand into the objective function
	\[	w_1z_1 + w_2 z_2 + ... + w_Lz_L.\]
Let's write it all out to see exactly what's happening.
\begin{align*}
		w_1z_1 & = w_1 \frac{ \dfrac{q}{A} \left[\dfrac{w_1}{\alpha_1}\right]^{1 / (\sigma - 1)} }{\left( \sum_{\ell=1}^L \alpha_{\ell} \left[ \dfrac{w_{\ell}}{\alpha_{\ell}} \right]^{\sigma / (\sigma - 1)} \right) ^{1 / \sigma}},\\
		w_2z_2 & = w_2 \frac{ \dfrac{q}{A} \left[\dfrac{w_2}{\alpha_2}\right]^{1 / (\sigma - 1)} }{\left( \sum_{\ell=1}^L \alpha_{\ell} \left[ \dfrac{w_{\ell}}{\alpha_{\ell}} \right]^{\sigma / (\sigma - 1)} \right) ^{1 / \sigma}},\\
			&\mathrel{\makebox[\widthof{=}]{\vdots}} \\
		w_Lz_L &= w_L \frac{\dfrac{q}{A} \left[\dfrac{w_L}{\alpha_L}\right]^{1 / (\sigma - 1)} }{\left( \sum_{\ell=1}^L \alpha_{\ell} \left[ \dfrac{w_{\ell}}{\alpha_{\ell}} \right]^{\sigma / (\sigma - 1)} \right) ^{1 / \sigma}}.
\end{align*}
Summing them together, we have
	\[ \frac{q}{A}  \frac{ \sum_{\ell=1}^L w_{\ell} \left[ \dfrac{w_{\ell}}{\alpha_{\ell}} \right]^{1 / (\sigma - 1)}}{\left( \sum_{\ell=1}^L \alpha_{\ell} \left[ \dfrac{w_{\ell}}{\alpha_{\ell}} \right]^{\sigma / (\sigma - 1)} \right) ^{1 / \sigma}}.\]
Now you should verify to yourself that the sum in the numerator and the sum in the denominator are actually the same thing. Thus, we have the cost function
	\[ c(q, w) = \frac{q}{A}  \left( \sum_{\ell=1}^L \alpha_{\ell} \left[ \frac{w_{\ell}}{\alpha_{\ell}} \right]^{\sigma / (\sigma - 1)} \right)^{(\sigma  - 1)/\sigma}.\]
Notice that this analysis requires that $\sigma \neq 0$ and $\sigma \neq 1$. The case where $\sigma \rightarrow 0$ actually corresponds to a Cobb-Douglas form  with constant returns to scale. The case where $\sigma=1$ is a linear function. I will discuss these cases later. 
	
	
	

\section{Output Supply Function and Profit Function}


\subsection{Output Supply Function and Profit Function}

Things won't pan out the way we might expect, but it is instructive to try anyway. We have the cost function, so we can try to find the output supply function by solving
	\[	\argmax_{q \geq 0} pq	 - \frac{q}{A}  \left( \sum_{\ell=1}^L \alpha_{\ell} \left[ \frac{w_{\ell}}{\alpha_{\ell}} \right]^{\sigma / (\sigma - 1)} \right)^{(\sigma  - 1)/\sigma}.	\]
Taking the derivative with respect to $q$ to find the critical point results in
	\[ p -  \frac{1}{A}  \left( \sum_{\ell=1}^L \alpha_{\ell} \left[ \frac{w_{\ell}}{\alpha_{\ell}} \right]^{\sigma / (\sigma - 1)} \right)^{(\sigma  - 1)/\sigma} :=0. \]
But um, there's no $q$ to solve for. What gives? It turns out that a CES function exhibits constant returns to scale:
\begin{align*}
	f(\lambda x) &= 	 A \left(\sum_{\ell = 1}^L \alpha_{\ell} [\lambda z_{\ell}]^{\sigma} \right)^{1 / \sigma} \\	
		& = A \left(\sum_{\ell = 1}^L \alpha_{\ell} \lambda^{\sigma} z_{\ell}^{\sigma} \right)^{1 / \sigma} \\	
		& = \lambda f(x).
\end{align*}

Recall that a production function exhibiting constant returns to scale either has a profit of zero or infinity. Looking at the profit maximizing condition, both terms are linear in $q$. Thus, we can have three situations:
\begin{itemize}
	\itemsep0em
	\item Revenue exceeds cost at any $q$,
	\item Revenue equals cost at any $q$,
	\item Revenue is less than cost at any $q$. 
\end{itemize}


\subsubsection{Price Exceeds Marginal Cost}
In the first case, the firm would produce an infinite amount $q= \infty$ since profit can only increase with each additional unit of output, and thus the firm would have infinite profit. This will be the case if 
	\[
		 p >  \frac{1}{A}  \left( \sum_{\ell=1}^L \alpha_{\ell} \left[ \frac{w_{\ell}}{\alpha_{\ell}} \right]^{\sigma / (\sigma - 1)} \right)^{(\sigma  - 1)/\sigma},
	\]
in other words, when price exceeds marginal cost. Intuitively, the firm can \emph{always} produce one more unit and sell it for a higher price than it cost to make it, and thus they will not stop producing. This also implies that the vector of inputs cannot be determined since there is no maximum profit and thus no profit maximizer. 


\subsubsection{Marginal Cost Exceeds Price}
The same logic implies that $q(p,w)=0$ and $\pi(p,w)=0$ when cost exceeds revenue, that is, when 
	\[
		 p <  \frac{1}{A}  \left( \sum_{\ell=1}^L \alpha_{\ell} \left[ \frac{w_{\ell}}{\alpha_{\ell}} \right]^{\sigma / (\sigma - 1)} \right)^{(\sigma  - 1)/\sigma},
	\]
in other words, when price is less then marginal cost. Intuitively, the firm can \emph{never} sell their product at a higher price than it cost to produce the product, so they're not even going to bother.  It should be clear that the vector of inputs in this case will simply be the zero vector.	
	
	
\subsubsection{Price Equals Marginal Cost}
The trickier case is when revenue equals cost, that is, when the price equals the marginal cost for \emph{every} level of output $q$. Profit will equal zero for any output $q$, and thus the producer is indifferent between any level of production. So they will produce at whatever level of output $\overline{q} \geq 0$ they feel like producing at, which gives a profit function of
	\[	\pi(p,w)= pq	 - \frac{q}{A}  \left( \sum_{\ell=1}^L \alpha_{\ell} \left[ \frac{w_{\ell}}{\alpha_{\ell}} \right]^{\sigma / (\sigma - 1)} \right)^{(\sigma  - 1)/\sigma}=0.	\]
We can use Hotelling's lemma to find the input demands, giving the function
\[
	 z_k(p,w) = \frac{  \dfrac{\overline{q}}{A} \left(\dfrac{w_k}{\alpha_k}\right)^{1 / (\sigma - 1)}}{ \left( \sum_{\ell=1}^L \alpha_{\ell} \left[ \dfrac{w_{\ell}}{\alpha_{\ell}} \right]^{\sigma / (\sigma - 1)} \right)^{1 / \sigma}}.
\]
\emph{This is exactly the conditional factor demand $z(q, w)$.} This should not be surprising, however. If the producer decides to produce $q$ output, then why would they demand anything other than the cost-minimizing inputs? Answer: they wouldn't. 



	
\section{Simple Case}
To make things simple, it is often assumed that $A=1$ and all $\alpha_{\ell}=1$. Furthermore, we can normalize $p=1$. So we are really considering the production function
	\[f(z)=(z_1^\sigma + z_2^\sigma)^{1 / \sigma}	\]
This allows us to write much nicer functions. Well, the conditional factor demand function isn't \emph{that} much nicer:
	\[ z_k(q,w)  =\frac{q w_k^{1 / (\sigma - 1)} }{\left( \sum_{\ell=1}^L  w_{\ell}^{\sigma / (\sigma - 1)} \right) ^{1 / \sigma}}.
	\]
Likewise, the cost function doesn't look a great deal different:
	\[ c(q, w) = q  \left( \sum_{\ell=1}^L w_{\ell} ^{\sigma / (\sigma - 1)} \right)^{(\sigma  - 1)/\sigma}.\]

We can, however, write much nicer forms of the profit function. Specifically, the marginal cost function is a lot easier to work with. The condition of price exceeding marginal cost, for example, can be written as 
	\[
		 \sum_{\ell=1}^L   w_{\ell} ^{\sigma / (\sigma - 1)} < 1.
	\]
So we can write the profit function as 
\[
\pi(w) = 
\begin{cases}
	\infty 	& 	\text{ if } \sum_{\ell=1}^L   w_{\ell} ^{\sigma / (\sigma - 1)} < 1,\\
	0 	& 	\text{ if } \sum_{\ell=1}^L   w_{\ell} ^{\sigma / (\sigma - 1)} \geq 1.\\
\end{cases}
\]
The output supply function will be
\[
q(w) = 
\begin{cases}
	\infty	& 	\text{ if } \sum_{\ell=1}^L   w_{\ell} ^{\sigma / (\sigma - 1)} < 1,\\
	\overline{q}	& 	\text{ if } \sum_{\ell=1}^L   w_{\ell} ^{\sigma / (\sigma - 1)} = 1.\\
	0 	& 	\text{ if } \sum_{\ell=1}^L   w_{\ell} ^{\sigma / (\sigma - 1)} > 1,\\
\end{cases}
\]
and the input demand functions will be
\[
z_k(w) = 
\begin{cases}
	\emptyset	& 	\text{ if } \sum_{\ell=1}^L   w_{\ell} ^{\sigma / (\sigma - 1)} < 1,\\
	\dfrac{\overline{q} w_k^{1 / (\sigma - 1)} }{\left( \sum_{\ell=1}^L  w_{\ell}^{\sigma / (\sigma - 1)} \right) ^{1 / \sigma}}	& 	\text{ if } \sum_{\ell=1}^L   w_{\ell} ^{\sigma / (\sigma - 1)} = 1.\\
	0 	& 	\text{ if } \sum_{\ell=1}^L   w_{\ell} ^{\sigma / (\sigma - 1)} > 1.\\
\end{cases}
\]


\section{Relationship to Other Functional Forms}	
The CES functional form can be seen as a generalization of a few different functional forms, in particular, Cobb-Douglas, Leontief, and linear. Perhaps the most obvious case is the one where $\sigma=1$, which gives rise to the linear form. 

\subsection{Linear Form}
When $\sigma=1$, the production function becomes
	\[	f(z)=A \left(\sum_{\ell = 1}^L \alpha_{\ell} z_{\ell} \right).\]
This is the case of perfect substitutes. Either we will have strict corner solutions or, in the unlikely case that the isocost slope is parallel to the constraint hyperplane, a continuum of solutions which could be either interior or corner. For more details, see the linear form notes. 

\subsection{Cobb-Douglas Form}
Let's immediately suppose that  $\sum_{\ell=1}^L \alpha_{\ell}=1$. Recall that CES exhibits constant returns to scale which, in the Cobb-Douglas case, requires $\sum_{\ell=1}^L \alpha_{\ell}=1$.  Turns out that as $\sigma$ approaches zero, the Cobb-Douglas function emerges. 

This shouldn't be too surprising. Recall that Cobb-Douglas functions exhibit a constant elasticity of substitution of $s=1$. Since $\sigma = (s-1)/s$, it follows that $\sigma=0$ implies $s=1$. So let's consider the limit
	\[	\lim_{\sigma \rightarrow 0} A \left(\sum_{\ell = 1}^L \alpha_{\ell} z_{\ell}^{\sigma} \right)^{1 / \sigma}.\]
	Your instinct might be to say, well $z_{\ell}^{\sigma}$ goes to 1, and the $\alpha_{\ell}$ terms will then just sum to 1, so as long as the exponent isn't going to zero (which it isn't), the answer should just be $A$... right? \emph{Wrong.} While it is true that $1^\infty=1$, this case is more subtle. In particular, it is not necessarily true that $f(x)^{g(x)} \rightarrow 1$ when $f(x) \rightarrow 1$ and $g(x) \rightarrow \infty$. To see this, notice that $\ln\big ( f(x)^{g(x)} \big) = g(x) \ln\big ( f(x) \big)$. So in the limit, we'll have $\infty \times 0$, which is indeterminate. As an concrete example, re-consider the limit derivation for $e$. Anyway, let's just throw away instinct for the time being. 

Being more analytical, let's take the logarithm of both sides to get
\begin{align*}
	\ln \big( f(z)/A \big) &= \ln \left[  \left(\sum_{\ell = 1}^L \alpha_{\ell} z_{\ell}^{\sigma} \right)^{1 / \sigma} \right]\\
		&=  \frac{  \ln \left[ \sum_{\ell = 1}^L \alpha_{\ell} z_{\ell}^{\sigma} \right]}{\sigma}.
\end{align*}
Again, in the limit this is indeterminate because we'll end up with $0/0$. But now it's in a form we can use L'Hopital's rule on. So take the derivative of the numerator and denominator with respect to $\sigma$, recalling that $dz^{\sigma}/d\sigma =z^{\sigma} \ln( z)$, to get
	\[\frac{\sum_{\ell =1}^L \alpha_{\ell} z_{\ell}^\sigma \ln(z_{\ell}) }{\sum_{\ell = 1}^L \alpha_{\ell} z_{\ell}^{\sigma}}.	\]
Taking the limit, we have
	\[ \lim_{\sigma \rightarrow 0} \ln\big( f(z)/A \big) = \frac{\sum_{\ell=1}^L \alpha_{\ell}\ln(z_{\ell}) }{1}= \sum_{\ell=1}^L \ln(z_{\ell}^{\alpha_{\ell}}).   \]
By exponentiating both sides with respect to $e$, we get exactly the Cobb-Douglas functional form,
	\[\lim_{\sigma \rightarrow 0} f(z) = A \sum_{\ell=1}^L z_{\ell}^{\alpha_{\ell}}. \]
	
	
\subsection{Leontief Form}	

Now we'll be taking the limit as $\sigma \rightarrow -\infty$. We'll begin with the same process that we used for the Cobb-Douglas case. In particular, let's start at the point where we've taken the logarithm of both sides and have applied L'Hopital's rule:
	\[\frac{\sum_{\ell =1}^L \alpha_{\ell} z_{\ell}^\sigma \ln(z_{\ell}) }{\sum_{\ell = 1}^L \alpha_{\ell} z_{\ell}^{\sigma}}.	\]
Define $z = \min\{z_1, ..., z_L\}$ and divide both numerator and denominator by $z^{\sigma}$ to get
	\[\frac{\sum_{\ell =1}^L \alpha_{\ell} \left(\dfrac{z_{\ell}}{z}\right)^\sigma \ln(z_{\ell}) }{\sum_{\ell = 1}^L \alpha_{\ell} \left(\dfrac{z_{\ell}}{z}\right)^{\sigma}}.	\]
This is useful because if $z_{\ell} > z$, then the fraction will go to zero in the limit; and if $z_{\ell}=z$, then the fraction will simply be 1 and remain 1 in the limit. Conveniently, every non-minimal term ${\ell}$ will drop out entirely. For simplicity, suppose $z_k=z$ is the unique minimum. Then the limit is
	\[	\lim_{\sigma \rightarrow -\infty} \ln \left( \frac{f(z)}{A} \right) = \frac{\alpha_{k} \ln(z_{k}) }{\alpha_k} = \ln(z_k). \]
	Exponentiating both sides with respect to $e$ gives
		\[ \lim_{\sigma \rightarrow -\infty} f(z) = Az_k = A \min\{z_1, ..., z_L \}.	\]
And thus we have a Leontief function. It is relatively straightforward to derive the same result if there are multiple $z_k=z_j = \min\{z_1, ..., z_L\}$. 


\end{document}
 
 
 