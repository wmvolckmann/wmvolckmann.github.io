\title{Basic LaTeX Walkthrough}
\author{WMV}
\date{10/05/2016}
\documentclass[12pt]{article}


\usepackage{amsmath}
\usepackage{amsfonts}
\usepackage{amssymb}
\usepackage{amsthm}
\usepackage{enumitem}



\newcommand{\C}{\mathbb{C}}
\newcommand{\F}{\mathbb{F}}
\newcommand{\N}{\mathbb{N}}
\newcommand{\Q}{\mathbb{Q}}
\newcommand{\R}{\mathbb{R}}
\newcommand{\Z}{\mathbb{Z}}
\newcommand{\Chi}{\mathcal{X}}


\begin{document}
\maketitle %this displays the title, author, date information from the preamble. Also, use % for comments. 


Before doing anything, try typesetting this TeX document. What that means is, it'll
turn all of this code into a PDF that hopefully looks the way you want it to look. In
MiKTeX the shortcut is control+t. 


\paragraph{Plain Text}

Anyway, this is how you type plain text. Nothing fancy, right? There are a few things
to note. First, it will only render one space. So if you use     a lot of spaces, most of
the spaces will disappear when you typeset. Similarly, if you enter a lot of empty
lines between two paragraphs, it will just start a new paragraph without a line 
between them. (Try it!)

If you want to \textbf{bold} something, you'll use the \verb|\textbf{}| command. 
To \emph{emphasize} something, use the \verb|\emph{}| command. (There is 
a \verb|\textit{}| command to italicize, but if you're trying to emphasize something,
use \verb|\emph{}| instead.) To \underline{underline} text, use \verb|\underline{}|. 

We can break up the document into different sections and subsections with
\verb|\section{}| and \verb|\subsection{}| commands. They'll automatically be
numbered for you unless you put an asterisk at the end, like \verb|\section|*\verb|{}|.
We can also title paragraphs with \verb|\paragraph{}|, as seen above. I'll start a 
new section right now.




\section{Math}


\subsection{In-Line}

We can put math directly into a paragraph, which is called \textbf{in-line} math. 
To do so, enter a dollar sign \$, then type in your math, then enter another \$ as 
soon as you want to use text again. For instance, \verb|$y=x^2$|  will output as 
$y=x^2$. If you don't like using the dollar signs, you can use \verb|\(y=x^2\)|, 
which will look exactly the same.  


\subsection{Display Environment}

For larger equations, you'll want to use \textbf{displayed} math. It will display the 
math on a new line below the paragraph you were just writing. This is also good 
if you want to show a short string of (in)equalities.  To do so, use the form 
\verb| \[y=x^2\]| and you'll get
		\[y=x^2\]


\subsection{Common Commands}

	\begin{itemize}
		\item For a times symbol, use \verb|\times|. For a dot product symbol, 
		use \verb|\dot|. 
		
		\item If you're in-line, write fractions as \verb|a/b|. If you're in display 
		mode, you can write fractions with \verb|\frac{a}{b}| to get
			\[ \frac{a}{b}. \]
		If you try the latter in-line, you'll get tiny fractions that are difficult to 
		read, e.g. $\frac{a}{b}$. 
		
		\item Use \verb|_| for subscripts and \verb|^| for superscripts. For 
		instance, $x_1^2$. If you want an entire expression in the subscript 
		or superscript (or generally for any command), put the entire expression
		in \verb|{}|. For instance, \verb|$x_1^{2y-5}$| will render as $x_1^{2y-5}$.
		
		\item Write square roots with \verb|\sqrt{}|. For instance, $\sqrt{-1}=i$. 
		
		\item Write strict inequalities with \verb|<| and \verb|>|, non-strict 
		with \verb|\leq| and \verb|\geq|. Note that you can express ``not 
		greater than'' by prefacing it with  \verb|\not|, for instance $5 \not > 6$. 
		(I think you can put \verb|\not| on pretty much anything, although there 
		is a special command \verb|neq| for $\neq$.)
		
		\item Most Greek letters are straightforward. \verb|\delta| will give you a
		$\delta$, and \verb|\Delta| will give you a $\Delta$. 
		
		\item For fancier letters, you'll usually use either the \verb|\mathbb{}| or 
		\verb|\mathcal{}| commands. For instance, \verb|\mathbb{R}| gives you
		the fancy real number symbol $\mathbb{R}$.
		
		\item Subsets and supersets are \verb|\subset| and \verb|\superset|, 
		respectively. Add \verb|eq| at the end to make them non-strict, e.g. 
		$A \subseteq B$. To take an arbitrary element from a set, use \verb|\in|, 
		so $x \in X$. Unions are given with \verb|\cup| and intersections are given 
		with \verb|\cap|, for instance,  
			\[	(A \cup B)^c=A^c \cap B^c . \]
		The empty set is simply \verb|\emptyset|. 
		
		\item Use \verb|\{ ... \}| for curly brackets. 
		
		\item For a partial derivative, use \verb|\partial|, e.g. $\partial z/\partial y$. 
		
		\item For limits, use \verb|\lim| with a subscript explaining what the limit is 
		doing, e.g.
				\[ \lim_{x \rightarrow \infty}  \]
		Infinity is given by \verb|\infty| and  the arrow is \verb|\rightarrow|. If you 
		try putting the limit in-line, it'll look kinda weird. 
		
		\item Use \verb|\sum| for sums. You'll usually want to try to use sums in the 
		display environment because they can be bulky, for instance
			\[\sum_{i=1}^n i = \frac{n(n+1)}{2}.\]
			
		\item Write integrals with \verb|\int| with sub and superscripts for the limit. 
		You'll want to put these in the display environment most of the time as well:
			\[ \int_{-\infty}^{\infty} \frac{1}{\pi(1+x^2)} \; dx = 1 .\]
			
		\item Sometimes you'll want really big parentheses. For instance,
				\[ ( \sum_{i=1}^n i)^2	\]
			just looks weird. In this case, put \verb|\left| before the left parenthesis 
			and \verb|\right| before the right parenthesis.:
				\[ \left( \sum_{i=1}^n i \right)^2	\]
				
		\item If you want to display a dollar sign or a percentage or some other symbol 
		that's used as code, try putting a \verb|\| before it. Then we can display \$ and 
		\& and \% and so forth.
	\end{itemize}
	
	
\subsection{Matrices}

Matrices are where things can get a little bit messy, if nothing else because matrices can be 
really big. Here's a simple $3 \times 3$ matrix. 
\[
	\begin{bmatrix}
		a	&	b	&	c	\\
		d	&	e	&	f	\\
		g	&	h	&	i	\\
	\end{bmatrix}
\]
In this case, we began a matrix \textbf{environment} which has a specific syntax for inputing
matrices. When done inputting, you end the environment so things return to normal. Use 
\verb|pmatrix| if you prefer your matrices in parenthesis, and use \verb|vmatrix| for 
determinants. Also note that \verb|\\| typically will start a new line in any environment.


\subsection{Lots of Math}
If we have a large collection of equations, we might want to write each equation as its own 
line. Furthermore, we might want each equation to be numbered. Do so with the \verb|gather|
environment. 
\begin{gather}
	y=mx+b	\\
	l=w \times h	\\
	x=-b \pm \frac{\sqrt{b^2 - 4ac}}{2a}
\end{gather}

Sometimes we might have a long chain of (in)equalities and we can't fit them all on one line.
In that case, use the \verb|align| environment. For example, 
\begin{align*}
	 P(D) &= P(D|C)P(C) + P(D|N)P(N) \\
	  	&= (0.90)(0.02) + (0.20)(0.98) \\ 
	  	&= 0.214.
\end{align*}
Each line is aligned by your placement of the \&.

	
\subsection{New Commands}
	
	Typing \verb|\mathbb{R}| for $\mathbb{R}$ over and over again gets old very quickly. 
	What we can do is create a new command that allows us to just type, say, \verb|\R| 
	instead. If you look in the preamble in this file, you'll see a handful of new commands
	I've made for this purpose. 
	
	
\section{Lists}

There are two types of lists: bullet lists and enumerated lists. Create a bullet list as follows:
\begin{itemize}	
	\item Every type you use \verb|\item|, you will create a new bullet point.
	
	\item See?
\end{itemize}
Or we might want to make a numbered list instead:
\begin{enumerate}
	\item Sometimes it would make more sense to number the lists.
	
	\item Like this. 
\end{enumerate}
We have to get a little fancy to have an alphabetized list, but it's not too bad:
\begin{enumerate}[label=\textbf{(\alph*)}]
	\item We just had to add that little label tag.
	
	\item I bolded it too just to illustrate that such a thing is possible.
\end{enumerate}
	
	
	\section{Other Stuff}
	
	Yeah, I'll add some other stuff as need warrants. 
 
	


\end{document}


	