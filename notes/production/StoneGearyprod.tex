\title{Stone-Geary Production Function}
\author{William M Volckmann II}
\documentclass[12pt]{article}
\usepackage{bm}
\usepackage{amsmath}
\usepackage{amsfonts}
\usepackage{graphicx}
\usepackage{amssymb}
\usepackage{amsthm}
\usepackage{setspace}
\usepackage{amsthm}
\usepackage{mathtools}
\usepackage{enumitem}
\usepackage{xifthen}
\usepackage{titlesec}
\usepackage[normalem]{ulem}
\usepackage[final]{pdfpages}
\usepackage[top=1.25in, left=1.25in, right=1.25in]{geometry}

\newcommand{\C}{\mathbb{C}}
\newcommand{\F}{\mathbb{F}}
\newcommand{\N}{\mathbb{N}}
\newcommand{\Q}{\mathbb{Q}}
\newcommand{\R}{\mathbb{R}}
\newcommand{\Z}{\mathbb{Z}}
\newcommand{\Chi}{\mathcal{X}}
\newcommand{\grad}{\nabla}
\newcommand{\B}{\beta}
\newcommand{\BH}{\hat{\beta}}
\newcommand{\bh}{\hat{\beta}}
\newcommand{\sumn}{\sum_{i=1}^n}
\newcommand{\crit}{c_{\alpha}}
\newcommand{\given}{\; | \;}
\newcommand{\xbar}{\bar{X}_n}
\newcommand{\asim}{\overset{a}{\sim}}
\newcommand{\Lindent}{\hspace{.4cm} \Longrightarrow \hspace{.4cm}}
\renewcommand{\vec}[1]{\mathbf{#1}}
\DeclareMathOperator*{\argmax}{arg\,max}
\DeclareMathOperator*{\argmin}{arg\,min}

\DeclareMathOperator*{\plim}{plim}
\DeclareMathOperator{\rank}{rank}

\newtheorem{theorem}{Theorem}
\theoremstyle{definition}
\newtheorem{definition}{Definition}
\newtheorem{example}{Example}

\setenumerate{itemsep=-1pt, label=\textbf{(\alph*)}}
%\setlength{\jot}{8pt}
%\setlength{\baselineskip}{1.5em}
%\titlespacing*{\section}{0pt}{4ex plus 1ex minus .2ex}{0ex plus .2ex}
%\titlespacing*{\subsection}{0pt}{4ex plus 1ex minus .2ex}{0ex plus .2ex}
%\titlespacing*{\subsubsection}{0pt}{3ex plus 1ex minus .2ex}{0ex plus .2ex}

\newcounter{ProbCounter}
\setcounter{ProbCounter}{1}
\newcommand{\problem}[1][]{%
\ifthenelse{\equal{#1}{}}{\section*{Problem \arabic{ProbCounter}}}
{\section*{Problem \arabic{ProbCounter} (#1)}}%
\stepcounter{ProbCounter}}



\begin{document}


\maketitle
\singlespace


We will consider a setting with $L+1$ commodities. The first $L$ commodities serve as inputs and the $L+1$th commodity is the output. Input commodity $\ell$ has input price $w_{\ell}$, and the output commodity has price $p$. We consider the general Stone-Geary production function
	\[	f(z)=A \prod_{{\ell}=1}^L (z_{\ell} - \gamma_{\ell})^{\alpha_{\ell}},\]
where $A$ represents the total-factor productivity, an exogenous measure of the level of ``technology." Notice how similar this is to Cobb-Douglas technology. The key difference are these $\gamma_{\ell}$ terms, which represent a sort of ``subsistence level'' below which the function is not defined.  In other words, it must be the case that $z_{\ell} \geq \gamma_{\ell}$. In the context of production, this means that at least $\gamma_{\ell}$ units of good ${\ell}$ must be used in production -- the interpretation is not as clear in production theory as it is in consumer theory. Perhaps the firm signed a contract for $\gamma_{\ell}$ units of input ${\ell}$ but have recently changed their production process so that they no longer need to use input $\ell$, and it would be too much of a hassle to try to sell them to someone else.



\section{Factor Demand and Cost Minimization}
The cost minimization problem is
	\[	\min_{z \geq 0} w_1z_1 + ... + w_{L}z_{L}  \;\; \text{ s.t. } \;\; A \prod_{\ell = 1}^{L} (z_{\ell} - \gamma_{\ell})^{\alpha_{\ell}} \geq q. \]	
We will find the factor demand functions first, and then utilize those to find the cost function. 
	

	
\subsection{Conditional Factor Demand Functions}
To find the conditional factor demands, we want to solve
	\[	\argmin_{z \geq 0} w_1z_1 + ... + w_{L}z_{L}  \;\; \text{ s.t. } \;\; A \prod_{\ell = 1}^{L} (z_{\ell} - \gamma_{\ell})^{\alpha_{\ell}} \geq q. 			\]	
Not only is an interior solution where all $z_{\ell}>0$ is necessary, but it also has to be the case that all $z_{\ell} > \gamma_{\ell}$. Otherwise we'll have some $z_{\ell} = \gamma_{\ell}$ and production will be zero, not satisfying the requirement of production of at least $q$. Thus, the Lagrangian is 
	\[	L(z, \lambda) = w_1z_1 + ... + w_Lz_L + \lambda\left[q -  A \prod_{\ell = 1}^{L} (z_{\ell} - \gamma_{\ell})^{\alpha_{\ell}} \right].	\]	
The first order conditions are
\begin{align}
	\frac{\partial L(z, \lambda)}{\partial z_k} = w_k - \lambda \frac{\alpha_{k}}{( z_{k} - \gamma_{k})}A \prod_{\ell = 1}^{L} (z_{\ell} - \gamma_{\ell})^{\alpha_{\ell}} &:=0, \label{cfdfoc1}\\
	 \lambda\left[q -  A \prod_{\ell = 1}^{L} (z_{\ell} - \gamma_{\ell})^{\alpha_{\ell}} \right] &:=0, \label{cfdfoc2}\\
 	z_{\ell} &> \gamma_{\ell},\\
 	\gamma_{\ell} &\geq 0.
\end{align}
We can use equation (\ref{cfdfoc1}) to show that
	\[\frac{w_{\ell} (z_{\ell} - \gamma_{\ell})}{\alpha_{\ell}}	 = \lambda f(z),	\]
and therefore we have the string of equalities
	\[\frac{w_{1} (z_{1} - \gamma_{1})}{\alpha_{1}} = \frac{w_2 (z_2 - \gamma_2)}{\alpha_2} = ... = \frac{w_{L} (z_{L} - \gamma_{L})}{\alpha_{L}}.		\]
Solve for $(z_2 - \gamma_2)^{\alpha_2}$ in terms of $z_1$ to get
	\[(z_2 - \gamma_2)^{\alpha_2} = \left(\frac{\alpha_2 w_1}{\alpha_1 w_2}\right)^{\alpha_2}	(z_1 - \gamma_1)^{\alpha_2} .\]
Doing this for all $z_{\ell}$, we have the system
\begin{align*}
	(z_1 - \gamma_1)^{\alpha_1} &= \left(\frac{w_1 \alpha_1}{\alpha_1 w_1} \right)^{\alpha_1}(z_1  - \gamma_1)^{\alpha_1} ,\\
	(z_2 - \gamma_2)^{\alpha_2} &= \left( \frac{ w_1 \alpha_2}{\alpha_1 w_2} \right)^{\alpha_2}(z_1  - \gamma_1)^{\alpha_2} ,\\
	&\mathrel{\makebox[\widthof{=}]{\vdots}} \\
	(z_L - \gamma_L)^{\alpha_L} &= \left( \frac{ w_1 \alpha_L}{\alpha_1 w_L} \right)^{\alpha_L}(z_1  - \gamma_1)^{\alpha_L}.
\end{align*}
Notice that from equation (\ref{cfdfoc1}), we can't have $\lambda=0$ because otherwise $w_k=0$ whereas we assume $w \gg 0$.  Therefore from equation (\ref{cfdfoc2}), we must have
	\[	A \prod_{\ell = 1}^{L} (z_{\ell} - \gamma_{\ell})^{\alpha_{\ell}} = q. \]
Thus, we can write
	\[A \left(\frac{w_1 \alpha_1}{\alpha_1 w_1} \right)^{\alpha_1}(z_1  - \gamma_1)^{\alpha_1}\left( \frac{ w_1 \alpha_2}{\alpha_1 w_2} \right)^{\alpha_2}(z_1  - \gamma_1)^{\alpha_2} \hdots \left( \frac{ w_1 \alpha_L}{\alpha_1 w_L} \right)^{\alpha_L}(z_1  - \gamma_1)^{\alpha_L} = q.	\]
Let $\sum_{\ell=1}^L \alpha_{\ell} = \alpha$. We can rewrite the preceding equation as
	\[A (z_1 - \gamma_1)^{\alpha} \left( \frac{w_1}{\alpha_1} \right)^a \prod_{\ell=1}^L \left(\frac{\alpha_{\ell}}{w_{\ell}}\right)^{\alpha_{\ell}} = q.	\]
Finally, we can solve for $z_1$ to get 
	\[ z_1 =   \frac{\alpha_1}{w_1} \left(\frac{q}{A}  \prod_{\ell=1}^L \left[\frac{w_{\ell}}{\alpha_{\ell}}\right]^{\alpha_{\ell}}\right)^{1 / \alpha} + \gamma_1. \]
Notice that this is exactly the Cobb-Douglas conditional factor demand but with $\gamma_1$ added. More generally,
	\[ z_k =   \frac{\alpha_k}{w_k} \left(\frac{q}{A}  \prod_{\ell=1}^L \left[\frac{w_{\ell}}{\alpha_{\ell}}\right]^{\alpha_{\ell}}\right)^{1 / \alpha} + \gamma_k. \]
	
	
	
\subsection{Cost Function}
To find the cost function, we can plug in the conditional factor demands into the objective function. Let's look more closely at the first term,
\begin{align*}
	w_1 z_1 &= w_1 \left[ \frac{\alpha_1}{w_1} \left(\frac{q}{A}  \prod_{\ell=1}^L \left[\frac{w_{\ell}}{\alpha_{\ell}}\right]^{\alpha_{\ell}}\right)^{1 / \alpha} + \gamma_1\right] \Lindent \alpha_1 \left(\frac{q}{A}  \prod_{\ell=1}^L \left[\frac{w_{\ell}}{\alpha_{\ell}}\right]^{\alpha_{\ell}}\right)^{1 / \alpha} + w_1\gamma_1.
\end{align*}
When we sum all of them up, we'll have the cost function
	\[c(w,q) = \alpha  \left(\frac{q}{A}  \prod_{\ell=1}^L \left[\frac{w_{\ell}}{\alpha_{\ell}}\right]^{\alpha_{\ell}}\right)^{1 / \alpha} + \sum_{\ell=1}^L w_{\ell}\gamma_{\ell}. \]
	
	

\section{Output Supply Function and Profit Function}


\subsection{Output Supply Function}

To find the output supply function, we want to solve
	\[\argmax_{q\geq0} pq -  \alpha  \left(\frac{q}{A}  \prod_{\ell=1}^L \left[\frac{w_{\ell}}{\alpha_{\ell}}\right]^{\alpha_{\ell}}\right)^{1 / \alpha} - \sum_{\ell=1}^L w_{\ell}\gamma_{\ell}.	\]
All we need to do is take the derivative with respect to $q$ and find the critical point. Note that we need $\alpha < 1$ to ensure that the function is concave in $q$ and thus the critical point is a maximum. Also notice that since there is no $q$ in the sum, we get the same result as with Cobb-Douglas:
	\[q(p,w) = \left(p^{\alpha} A \prod_{\ell=1}^L \left[ \frac{\alpha_{\ell}}{w_{\ell}} \right]^{\alpha_{\ell}} \right)^{1 / (1 - \alpha)}.	\]
	
\subsection{Profit Function} 
Now for the profit function, we plug the output supply function into the objective function. Again, it's going to be practically identical to the Cobb-Douglas case except with the sum tacked on at the end:
	\[ \pi(p,w) = (1 - \alpha) \left( pA \prod_{\ell=1}^L \left[ \frac{\alpha_{\ell}}{w_{\ell} }\right]^{\alpha_{\ell}} \right)^{1 / (1 - \alpha)} +  \sum_{\ell=1}^L w_{\ell}\gamma_{\ell}. \]
	

\subsection{Input Demand Functions}
We can appeal to Cobb-Douglas again. Using Shepherd's lemma, we find the input demand functions to be
	\[z_{k}(p,w) =-\frac{\partial \pi(p,w)}{\partial w_k} = \frac{\alpha_k}{w_k}	 \left( pA \prod_{\ell=1}^L \left[ \frac{\alpha_{\ell}}{w_{\ell} }\right]^{\alpha_{\ell}} \right)^{1 / (1 - \alpha)} + \gamma_k. \]






 \end{document}
 