\title{ECN 200B---Arrow-Debreu Proof}
\author{William M Volckmann II}
\documentclass[12pt]{article}
\usepackage{bm}
\usepackage{amsmath}
\usepackage{amsfonts}
\usepackage{mathrsfs}
\usepackage{graphicx}
\usepackage{amssymb}
\usepackage{amsthm}
\usepackage{setspace}
\usepackage{amsthm}
\usepackage{mathtools}
\usepackage{thmtools}
\usepackage{enumitem}
\usepackage{xifthen}
\usepackage{titlesec}
\usepackage[normalem]{ulem}
\usepackage[final]{pdfpages}
\usepackage{pgfplots}
\usetikzlibrary{positioning}
%\usepackage[top=1.25in, left=1.25in, right=1.25in]{geometry}

%Vectors and metrics
\newcommand{\norm}[1]{\left\Vert{#1}\right\Vert}
\newcommand{\abs}[1]{\left\vert{#1}\right\vert}
\renewcommand{\a}{\vec{a}}
\renewcommand{\b}{\vec{b}}
\renewcommand{\c}{\vec{c}}
\newcommand{\p}{\vec{p}}
\newcommand{\q}{\vec{q}}
\renewcommand{\r}{\vec{r}}
\renewcommand{\u}{\vec{u}}
\renewcommand{\v}{\vec{v}}
\newcommand{\w}{\vec{w}}
\newcommand{\x}{\vec{x}}
\newcommand{\y}{\vec{y}}
\newcommand{\z}{\vec{z}}

\newcommand{\C}{\mathbb{C}}
\newcommand{\F}{\mathbb{F}}
\newcommand{\N}{\mathbb{N}}
\newcommand{\Q}{\mathbb{Q}}
\newcommand{\R}{\mathbb{R}}
\newcommand{\Z}{\mathbb{Z}}
\newcommand{\Chi}{\mathcal{X}}
\newcommand{\grad}{\nabla}
\renewcommand{\l}{\ell}
\renewcommand{\vec}[1]{\mathbf{#1}}
\DeclareMathOperator*{\argmax}{arg\,max}
\DeclareMathOperator*{\argmin}{arg\,min}


%Statistics and Probability 
\DeclareMathOperator*{\var}{Var}
\DeclareMathOperator*{\cov}{Cov}
\DeclareMathOperator*{\plim}{plim}
\DeclareMathOperator*{\supp}{supp}
\newcommand{\cprob}{\overset{p}{\rightarrow}}
\newcommand{\cdist}{\overset{d}{\rightarrow}}
\newcommand{\normal}[2]{\mathcal{N} \left({#1}, {#2} \right)}
\newcommand{\BH}{\hat{\beta}}
\newcommand{\bh}{\hat{\beta}}
\newcommand{\sumn}{\sum_{i=1}^n}
\newcommand{\sumi}{\sum_{i=1}^I}
\newcommand{\crit}{c_{\alpha}}
\newcommand{\given}{\; | \;}
\newcommand{\xbar}{\bar{X}_n}
\newcommand{\asim}{\overset{a}{\sim}}
\renewcommand{\vec}[1]{\mathbf{#1}}

\newcommand{\Lindent}{\hspace{.2cm} \Longrightarrow \hspace{.4cm}}


\newtheorem{theorem}{Theorem}
\newtheorem*{theorem*}{Theorem}
\declaretheorem[style=definition, qed=$//$]{claim}
\theoremstyle{definition}
\newtheorem{definition}{Definition}
\declaretheorem[style=definition,qed=$\blacksquare$]{example}

\setenumerate{itemsep=-1pt, label=\textbf{(\alph*)}}
\setitemize{itemsep=-1pt}

\begin{document}


\maketitle
\onehalfspace




\begin{theorem*}[Arrow-Debreu Theorem]
	Fix an exchange economy. If $u^i$ is continuous, strictly monotone, and strictly quasiconcave; and if each $w^i \gg 0$; then there exists a competitive equilibrium.
\end{theorem*}


\paragraph*{The Setup.}

Normalize prices to the simplex $\Delta$. We will refer to the interior of the simplex as 
	\[\Delta^o = \Delta \cap \R^L_{++} = \left \{ p \in \R^L_{++} \Bigg\vert \sum_{\ell=1}^L p_{\ell} =1 \right \},	\]	
and the boundary of the simplex as
	\[\Delta^{\partial}  = \Delta \setminus \Delta^o.	\]

For any $i$ and any $p \in \Delta^o$, let $x^i(p)$ maximize individual $i$'s utility subject to $p \cdot x \leq p \cdot w^i$. Define the excess aggregate demand in the usual way,
	\[z(p)=\sumi 	x^i(p) - w^i.\]
	


\paragraph{Properties of Excess Demand.}

\begin{claim}
	\emph{$z(p)$ is a continuous function.} Since utility is assumed to be strictly quasiconcave, it follows that $x^i(p)$ is a function, and therefore so is $z(p)$. Because utility is also assumed to be continuous, it follows that $x^i(p)$ is continuous\footnote{By the Theorem of the Maximum, which states that an optimized function is continuous as its parameter changes, in this case $p$, under certain conditions.}, and therefore $z(p)$ is continuous as well. 
\end{claim}

\begin{claim}
	\emph{For any $p \in \Delta^o$, $p\cdot z(p)=0$.} Since utility is strictly monotone, it is locally nonsatiated, and therefore the budget constraint is an equality, that is, $x^i(p)=w^i(p)$ for all $i$. It follows that $p \cdot [x^i(p) - w^i(p)]=0$ for all $i$. And therefore
		\[p \cdot \sumi [x^i(p) - w^i ]= p \cdot z(p) = 0.	\qedhere \]
\end{claim}


One problem arises when $p \in \Delta^{\partial}$, however. Specifically, if $p_L=0$, then $x^i(p)_L$ will be undefined---strongly monotone preferences and strictly positive endowments means everyone will want to buy an infinite number of good $L$. So we'll need to consider boundary prices and interior prices separately, but not \emph{too} separately. I'll explain in a bit. 


\paragraph{The Gamma Correspondence.} I'll illustrate the idea before defining anything. Suppose $p \in \Delta^o$, and  $z(p)=(3, 1, 1, -1)$. Let's solve
	\[	\max_{\delta_1, \delta_2, \delta_3, \delta_4} 3\delta_1 + 1 \delta_2 + 1 \delta_3 - 1 \delta_4\]
	such that $\sum_{j=1}^4 \delta_j=1$. Clearly we would put all of the weight on $\delta_1$ because it is the largest positive number; that is, have $\delta_1=1$, the other $\delta_{j \neq 1}  = 0$. So the idea is, whichever good has the highest excess demand, we should set its price to 1 and every other price to zero. Let $\Gamma(p)$ be the maximizing set of $\delta_j$, in this case, $\Gamma(p)=[1,0,0,0]$. 

Now suppose that  $p \in \Delta^{\partial}$. In this case, excess demand is not defined. What we'll do is take all possible vectors  $\gamma \in \Delta$ that satisfy $p \cdot \gamma =0$. For instance, if $p=( .5, .5, 0, 0)$, then $\delta_1=\delta_2=0$, and $\delta_3+\delta_4 = 1$, so $\delta = (0, 0, .5, .5)$ would work, or $(0, 0, .1, .9)$. 

Thus the correspondence we'll be working with is
\[
	\Gamma(p) = 
	\begin{cases}
		\argmax_{\gamma \in \Delta} z(p) \cdot \gamma &\text{if } p \in \Delta^0,\\
		\{\gamma \in \Delta | p \cdot \gamma =0\} &\text{if } p \in \Delta^{\partial}.
	\end{cases}
\]
An important point to note here is that if $p \in \Delta^o$ and $z(p) \neq (0, \hdots, 0)$, then $\Gamma(p) \subseteq \Delta^{\partial}$. And thus by the contrapositive, if $\Delta(p) \not\subseteq \Delta^{\partial}$ and $p \in \Delta^*$, then $z(p)=(0, \hdots, 0)$ and we are done. 

Eventually we want to invoke Kakutani's theorem on $\Gamma(p)$, but of course $\Gamma(p)$ must satisfy certain conditions to justify doing so. In the interest of time, we will take for granted a few  different properties, although each can be shown. 
\begin{enumerate}
	\item We need $\Gamma$ to map $\Delta \twoheadrightarrow \Delta$. This condition is rather obvious since each $\gamma$ is taken from $\Delta$. 
	\item $\Gamma$ is nonempty, compact, and convex-valued.
	\item $\Gamma$ is upper-hemicontinuous for $p \in \Delta^o$. 
\end{enumerate}
Let's show upper-hemicontinuity. Consider a sequence of prices $(p_n)_{n=1}^{\infty} \in \Delta^o$ that converges to positive prices in all except one $\bar{p}_L=0$, that is,
	\[ \left(p_{1(n)}, \hdots , p_{L-1(n)}, p_{L(n)} \right) \rightarrow  (\bar{p}_1, \hdots , \bar{p}_{L-1}, 0)  = \bar{\delta} \in \Delta^{\partial}.\]
Let $\delta_n \in \Gamma(p_n)$ for any $n$. Our question is this: if $p_n$ converges to $\bar{p}$, will the correspondence converge to some $\gamma \in \Gamma(\bar{p})$?

We know that $z_L(p_n) \rightarrow \infty$. We also know that $p_{\ell(n)}>0$ and $\bar{p}_{\ell}>0$ for any $\ell \neq L$. Thus, for large enough $n$, we'll have 
	\[ z_L(p_n) > z_{\ell}(p_n).	\]
Which means for large enough $n$, we'll have $\Gamma(p_n) = \{0, 0, \hdots, 0, 1\}$. Therefore $\gamma_n \rightarrow (0, 0, \hdots, 0, 1) = \bar{\gamma}$. Furthermore, $\bar{p} \cdot \bar{\gamma} = 0$. So $\bar{\gamma} \in \Gamma(\bar{p})$. Hemicontinuity is established.

This is good---even though we're treating boundary prices and interior prices with separate correspondences, the case correspondence $\Gamma(p)$ is still upper-hemicontinuous.



\paragraph{The Fixed Point.} Okay great, so we can apply Kakutani's theorem---there exists some $p^*\in \Delta$ such that $p^* \in \Gamma(p^*)$. What can we say about $p^*$? 

The most important thing we can say is that it's not in the boundary. To see why, suppose $p^* =(1, 1, 0)$. Then we'll have $\delta = (0, 0, 1) \in \Gamma(p^*)$. But then if we take $\Gamma(\delta)$, we'll get, among other choices, $\Gamma(\delta) = (.5, .5, 0) \notin  \Gamma(p^*)$. 

More generally, suppose $p^*$ has $p_L^*=0$. Then the correspondence  $\Gamma(p^*)$ will consist of vectors with $\delta_L > 0$. So when we plug $\delta$ back into the correspondence with  $\Gamma(\delta)$, it will return vectors with $\tilde{\delta}_L=0$, which cannot be in $\Gamma(p^*)$ because  $\Gamma(p^*)$ consists of vectors with $\delta_L>0$. 

Since we know there exists some fixed point $p^*$, it follows that it must be in the interior. Furthermore, we know that if $p^* \in \Delta^o$ and $z(p) \neq (0, \hdots, 0)$, then $\Gamma(p^*) \subseteq \Delta^{\partial}$. By the contrapositive, the fact that $\Gamma(p^*) \not\subseteq \Delta^{\partial}$ means that $z(p^*)=(0, \hdots, 0)$. And thus we know that the equilibrium exists and has strictly positive prices on all commodities. 

\end{document}