\title{Basic LaTeX Walkthrough}
\author{WMV}
\date{10/05/2016}
\documentclass[12pt]{article}



\usepackage{amsmath}
\usepackage{amsfonts}
\usepackage{amssymb}
\usepackage{amsthm}



\newcommand{\C}{\mathbb{C}}
\newcommand{\F}{\mathbb{F}}
\newcommand{\N}{\mathbb{N}}
\newcommand{\Q}{\mathbb{Q}}
\newcommand{\R}{\mathbb{R}}
\newcommand{\Z}{\mathbb{Z}}
\newcommand{\Chi}{\mathcal{X}}



\begin{document}
\maketitle


Before doing anything, try typesetting this TeX document. What that means is, it'll turn all of this code into a PDF that hopefully looks the way you want it to look. In MiKTeX the shortcut is control+t. 

Anyway, this is how you type plain text. Nothing fancy, right? There are a few things to note. First, it will only render one space. So if you use     a lot of spaces, most of the spaces will disappear when you typeset. Similarly, if you enter a lot of empty lines between two paragraphs, it will just start a new paragraph without a line between them. (Try it!)

If you want to \textbf{bold} something, you'll write use the \verb|\textbf |command. To \emph{emphasize} something, use the \verb|\emph| command. (There is a \verb|\textit| command to italicize, but if you're trying to emphasize something, use \verb|emph| instead.) To \underline{underline} text, use \verb|\underline|. 

We can break up the document into different sections and subsections with \verb|\section| and \verb|\subsection| commands. They'll automatically be numbered for you unless you put an asterisk in front of them, like \verb|\section|*. We can also title paragraphs with \verb|\paragraph|. For instance, I'll start a new section right now.




\section{Math}


\subsection{In-Line}

We can put math directly into a paragraph, which is called \textbf{in-line} math. To do so, enter a dollar sign \$, then type in your math, then enter another \$ as soon as you want to use text again. For instance, \verb|$y=x^2$|  will output as $y=x^2$. If you don't like using the dollar signs, you can use \verb|\(y=x^2\)|, which will look exactly the same.  


\subsection{Display Environment}

For larger equations, you'll want to use \textbf{displayed} math. It will display the math on a new line below the paragraph you were just writing. This is also good if you want to show a short string of (in)equalities.  To do so, use the form  \verb| \[y=x^2\]|, and you'll get
		\[y=x^2\]


\subsection{Common Syntax}

	\begin{itemize}
		\item For a times symbol, use \verb|\times|. For a dot product symbol, use \verb|\dot|. 
		\item If you're in-line, write fractions as \verb|a/b|. If you're in display mode, you can write fractions with \verb|\frac| as \[ \frac{a}{b}. \]
		If you try the former in-line, you'll get tiny fractions that are difficult to read, e.g. $\frac{a}{b}$. 
		\item Use \verb|_| for subscripts and \verb|^| for superscripts. For instance, $x_1^2$. If you want an entire expression in the subscript or superscript (or generally for any command), put the entire express in \verb|{}|. For instance, $x_1^{2y-5}$. 
		\item Write square roots with \verb|\sqrt|. For instance, $\sqrt{-1}=i$. 
		\item Write strict inequalities with \verb|<| and \verb|>|, non-strict with \verb|\leq| and \verb|\geq|. Note that you can express ``not greater than'' by prefacing it with  \verb|\not|, for instance $5 \not > 6$. (I think you can put \verb|\not| on pretty much anything, although there is a special command \verb|neq| for $\neq$.)
		\item Most Greek letters are straightforward. \verb|\delta| will give you a $\delta$, and \verb|\Delta| will give you a $\Delta$. 
		\item For fancier letters, you'll usually use either the \verb|\mathbb| or \verb|\mathcal| commands. For instance, \verb|\mathbb{R}| gives you the fancy real number symbol $\mathbb{R}$.
		\item Subsets and supersets are \verb|\subset| and \verb|\superset|, respectively. Add \verb|eq| at the end to make them non-strict, e.g. $A \subseteq B$. To take an arbitrary element from a set, use \verb|in|, so $x \in X$. Unions are given with \verb|\cup| and intersections are given with \verb|\cap|, for instance,  
			\[	(A \cup B)^c=A^c \cap B^c . \]
		The empty set is simply \verb|\emptyset|. 
		\item For a partial derivative, use \verb|\partial|, e.g. $\partial z/\partial y$. 
		\item For limits, use \verb|\lim| with a subscript explaining what the limit is doing, e.g.
			\[ \lim_{x \rightarrow \infty}  \]
			Infinity is given by \verb|\infty| and  the arrow is \verb|\rightarrow|. If you try putting the limit in-line, it'll look kinda weird. 
		\item Use \verb|\sum| for sums. You'll usually want to try to use sums in the display environment because they can be bulky, for instance
			\[\sum_{i=1}^n i = \frac{n(n+1)}{2}.\]
		\item Write integrals with \verb|\int| with sub and superscripts for the limit. You'll want to put these in the display environment most of the time as well:
			\[ \int_{-\infty}^{\infty} \frac{1}{\pi(1+x^2)} \; dx = 1 .\]
	\end{itemize}

	
	\subsection{New Commands}
	
	Typing \verb|\mathbb{R}| for $\mathbb{R}$ gets old very quickly. What we can do is create a new command that allows us to just type, say, \verb|\R| instead. If you look in the preamble in this file, you'll see a handful of new commands I've made for this purpose. 
	


\end{document}


	