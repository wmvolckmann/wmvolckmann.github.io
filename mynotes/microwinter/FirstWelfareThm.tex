\title{ECN 200B---First Welfare Theorem Proof}
\author{William M Volckmann II}
\documentclass[12pt]{article}
\usepackage{bm}
\usepackage{amsmath}
\usepackage{amsfonts}
\usepackage{mathrsfs}
\usepackage{graphicx}
\usepackage{amssymb}
\usepackage{amsthm}
\usepackage{setspace}
\usepackage{amsthm}
\usepackage{mathtools}
\usepackage{thmtools}
\usepackage{enumitem}
\usepackage{xifthen}
\usepackage{titlesec}
\usepackage[normalem]{ulem}
\usepackage[final]{pdfpages}
\usepackage{pgfplots}
\usetikzlibrary{positioning}
\usepackage[top=.6in]{geometry}

%Vectors and metrics
\newcommand{\norm}[1]{\left\Vert{#1}\right\Vert}
\newcommand{\abs}[1]{\left\vert{#1}\right\vert}
\renewcommand{\a}{\vec{a}}
\renewcommand{\b}{\vec{b}}
\renewcommand{\c}{\vec{c}}
\newcommand{\p}{\vec{p}}
\newcommand{\q}{\vec{q}}
\renewcommand{\r}{\vec{r}}
\renewcommand{\u}{\vec{u}}
\renewcommand{\v}{\vec{v}}
\newcommand{\w}{\vec{w}}
\newcommand{\x}{\vec{x}}
\newcommand{\y}{\vec{y}}
\newcommand{\z}{\vec{z}}

\newcommand{\C}{\mathbb{C}}
\newcommand{\F}{\mathbb{F}}
\newcommand{\N}{\mathbb{N}}
\newcommand{\Q}{\mathbb{Q}}
\newcommand{\R}{\mathbb{R}}
\newcommand{\Z}{\mathbb{Z}}
\newcommand{\Chi}{\mathcal{X}}
\newcommand{\grad}{\nabla}
\renewcommand{\l}{\ell}
\renewcommand{\vec}[1]{\mathbf{#1}}
\DeclareMathOperator*{\argmax}{arg\,max}
\DeclareMathOperator*{\argmin}{arg\,min}


%Statistics and Probability 
\DeclareMathOperator*{\var}{Var}
\DeclareMathOperator*{\cov}{Cov}
\DeclareMathOperator*{\plim}{plim}
\DeclareMathOperator*{\supp}{supp}
\newcommand{\cprob}{\overset{p}{\rightarrow}}
\newcommand{\cdist}{\overset{d}{\rightarrow}}
\newcommand{\normal}[2]{\mathcal{N} \left({#1}, {#2} \right)}
\newcommand{\BH}{\hat{\beta}}
\newcommand{\bh}{\hat{\beta}}
\newcommand{\sumn}{\sum_{i=1}^n}
\newcommand{\sumi}{\sum_{i=1}^I}
\newcommand{\sumj}{\sum_{j=1}^J}
\newcommand{\crit}{c_{\alpha}}
\newcommand{\given}{\; | \;}
\newcommand{\xbar}{\bar{X}_n}
\newcommand{\asim}{\overset{a}{\sim}}
\renewcommand{\vec}[1]{\mathbf{#1}}

\newcommand{\Lindent}{\hspace{.2cm} \Longrightarrow \hspace{.4cm}}


\newtheorem{theorem}{Theorem}
\newtheorem*{lemma}{Lemma}
\newtheorem*{theorem*}{Theorem}
\declaretheorem[style=definition, qed=$//$]{claim}
\theoremstyle{definition}
\newtheorem*{claim*}{Claim}
\newtheorem{definition}{Definition}
\declaretheorem[style=definition,qed=$\blacksquare$]{example}

\setenumerate{itemsep=-1pt, label=\textbf{(\alph*)}}
\setitemize{itemsep=-1pt}

\begin{document}


\maketitle
\onehalfspace

\section*{Preliminary Results}

 To begin with, we'll start with a lemma and a theorem.\footnote{No, I do not know why one qualifies as a lemma and the other a theorem.}

\begin{lemma}
Suppose that $u^i(\cdot)$ is locally nonsatiated and $x^*$ maximizes $u(x)$ subject to $px \leq m$. Then $u(x') \geq u(x^*)$ implies that $px' \geq m$. 
\end{lemma}
\begin{proof}
	Suppose there exists some $\hat{x}$ such that $u(\hat{x}) \geq u(x^*)$ and $p\hat{x} < m$. Because preferences are locally nonsatiated, we know that we can construct an $\epsilon$-ball around $\hat{x}$ and inside such a ball exists some $x'$  such that $u(x') > u(\hat{x})$, and furthermore we can choose $\epsilon$ so that $p x' \leq m$. This means that $x'$ is affordable and is preferred to $x^*$, meaning that $x^*$ is not actually the maximizer. This is a contradiction and the lemma is established. 
\end{proof}


\begin{theorem*}
	Suppose that $x^*$ maximizes $u(x)$ subject to $px \leq m$. Then $u(x) > u(x^*)$ implies $px > m$. 
\end{theorem*}
\begin{proof}
	Note that local nonsatiation has not been assumed. But in any case, this follows directly from the definition of a maximizer. If $x$ is preferred to $x^*$ and is also affordable, then $x^*$ isn't even the maximizer. Thus it has to be the case that $x$ is unaffordable. 
\end{proof}


\section*{First Fundamental Theorem of Welfare Economics}

\begin{theorem*}
	Fix a production economy, 
		\[			\{ I, J, (Y_j)_{j\in J}, (u^i, w^i, s^{i,j})_{i \in I, j \in J}\}. 	\]
	Suppose that all $u^i(\cdot)$ are locally nonsatiated. If $(p,x,y)$ is a competitive equilibrium, then $(x,y)$ is Pareto efficient.
\end{theorem*}

\paragraph{The Setup.} Suppose that $(p,x,y)$ is a competitive equilibrium but $(x,y)$ is not Pareto efficient. Since it is a competitive equilibrium, we know that
\begin{enumerate}
	\item For all $i$, $x^i$ maximizes $u^i(\tilde{x})$ subject to $p \tilde{x} \leq p w^i  + \sum_{j \in J} s^{i,j} p \cdot y^j$.
	\item For any $j$, $y^j$ maximizes $p \cdot \tilde{y}$ subject to $\tilde{y} \in Y^j$. 
	\item $\sum_{i \in I} x^i = \sum_{i \in I}w^i + \sum_{j \in J} y^j$. 
\end{enumerate}
And because the allocation is not Pareto efficient, there exists some allocation $(\hat{x}, \hat{y})$ such that 
\begin{enumerate}[label= \textbf{(\roman*)}]
	\item for all $i$, $\hat{x}^i \in \R^L_+$, 
	\item for all $j$, $\hat{y}^j \in Y^j$,
	\item $\sumi \hat{x}^i = \sumi w^i + \sumj \hat{y}^j$,
	\item for all $i$, $u^i(\hat{x}^i) \geq u^i(x^i)$, 
	\item for some $i^*$, $u^{i^*}(\hat{x}^{i^*}) > u^i(x^i)$.
\end{enumerate}

\paragraph{The Proof.}
By the previous theorem, it follows that $x^{i^*}$ must not be affordable, that is,
	\[p \hat{x}^{i^*} > p w^i  + \sum_{j \in J} s^{i,j} p \cdot y^j.\]
For any of the other individuals, by the previous lemma we must have
	\[p \hat{x}^{i} \geq p w^i  + \sum_{j \in J} s^{i,j} p \cdot y^j,\]	
and therefore for all $i$, we have 
	\[p \hat{x}^{i} \geq p w^i  + \sum_{j \in J} s^{i,j} p \cdot y^j.\]
Summing over all $i$ and taking prices out of the sums, we have
	\[p \sumi \hat{x}^{i} > p \sumi w^i  + p \sum_{i \in I,j \in J} s^{i,j} \cdot y^j.\]	
When we sum over the shares in firm $j$, each $\sumi s^{i,j}=1$, and therefore
\begin{equation}
	p \sumi \hat{x}^{i} > p \sumi w^i  + p \sum_{j \in J} y^j.
\end{equation}

	Because $y^j$ was assumed to be the (feasible) profit maximizer, it must be the case for any $j$ that $p \cdot \hat{y}^j \leq p\cdot y^j$. So if we sum over all $j$ and take prices out the sums, we get
	\[p \cdot \sumj \hat{y}^j \leq p\cdot \sumj y^j.	\]
Add the aggregate nominal value of endowments to both sides for 
	\[p \cdot \sumi w^i + p \cdot \sumj \hat{y}^j  \leq  p \cdot \sumi w^i + p\cdot \sumj y^j  .	\]
Combining this with equation (1), it follows that
		\[	p \cdot \sumi \hat{x}^i > p \cdot \sumi w^i + p \cdot \sumj \hat{y}^j,	\]
which contradicts point (iii) above. Thus the competitive equilibrium allocation must be Pareto efficient. \qed

\end{document}