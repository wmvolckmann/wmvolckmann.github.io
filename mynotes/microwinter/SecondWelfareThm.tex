\title{ECN 200B---Second Welfare Theorem Proof}
\author{William M Volckmann II}
\documentclass[12pt]{article}
\usepackage{bm}
\usepackage{amsmath}
\usepackage{amsfonts}
\usepackage{mathrsfs}
\usepackage{graphicx}
\usepackage{amssymb}
\usepackage{amsthm}
\usepackage{setspace}
\usepackage{amsthm}
\usepackage{mathtools}
\usepackage{thmtools}
\usepackage{enumitem}
\usepackage{xifthen}
\usepackage{titlesec}
\usepackage[normalem]{ulem}
\usepackage[final]{pdfpages}
\usepackage{pgfplots}
\usetikzlibrary{positioning}
\usepackage[top=.6in]{geometry}

%Vectors and metrics
\newcommand{\norm}[1]{\left\Vert{#1}\right\Vert}
\newcommand{\abs}[1]{\left\vert{#1}\right\vert}
\renewcommand{\a}{\vec{a}}
\renewcommand{\b}{\vec{b}}
\renewcommand{\c}{\vec{c}}
\newcommand{\p}{\vec{p}}
\newcommand{\q}{\vec{q}}
\renewcommand{\r}{\vec{r}}
\renewcommand{\u}{\vec{u}}
\renewcommand{\v}{\vec{v}}
\newcommand{\w}{\vec{w}}
\newcommand{\x}{\vec{x}}
\newcommand{\y}{\vec{y}}
\newcommand{\z}{\vec{z}}

\newcommand{\C}{\mathbb{C}}
\newcommand{\F}{\mathbb{F}}
\newcommand{\N}{\mathbb{N}}
\newcommand{\Q}{\mathbb{Q}}
\newcommand{\R}{\mathbb{R}}
\newcommand{\Z}{\mathbb{Z}}
\newcommand{\Chi}{\mathcal{X}}
\newcommand{\grad}{\nabla}
\renewcommand{\l}{\ell}
\renewcommand{\vec}[1]{\mathbf{#1}}
\DeclareMathOperator*{\argmax}{arg\,max}
\DeclareMathOperator*{\argmin}{arg\,min}


%Statistics and Probability 
\DeclareMathOperator*{\var}{Var}
\DeclareMathOperator*{\cov}{Cov}
\DeclareMathOperator*{\plim}{plim}
\DeclareMathOperator*{\supp}{supp}
\newcommand{\cprob}{\overset{p}{\rightarrow}}
\newcommand{\cdist}{\overset{d}{\rightarrow}}
\newcommand{\normal}[2]{\mathcal{N} \left({#1}, {#2} \right)}
\newcommand{\BH}{\hat{\beta}}
\newcommand{\bh}{\hat{\beta}}
\newcommand{\sumn}{\sum_{i=1}^n}
\newcommand{\sumi}{\sum_{i=1}^I}
\newcommand{\sumj}{\sum_{j=1}^J}
\newcommand{\crit}{c_{\alpha}}
\newcommand{\given}{\; | \;}
\newcommand{\xbar}{\bar{X}_n}
\newcommand{\asim}{\overset{a}{\sim}}
\renewcommand{\vec}[1]{\mathbf{#1}}

\newcommand{\Lindent}{\hspace{.2cm} \Longrightarrow \hspace{.4cm}}


\newtheorem{theorem}{Theorem}
\newtheorem*{lemma}{Lemma}
\newtheorem*{theorem*}{Theorem}
\declaretheorem[style=definition, qed=$//$]{claim}
\theoremstyle{definition}
\newtheorem{definition}{Definition}
\declaretheorem[style=definition,qed=$\blacksquare$]{example}

\setenumerate{itemsep=-1pt, label=\textbf{(\alph*)}}
\setitemize{itemsep=-1pt}

\begin{document}


\maketitle
\onehalfspace


\section*{Preliminary Results}

\begin{theorem}[Separating Hyperplane Theorem]
	Suppose sets $Q, Q' \in \R^A$ are disjoint and convex. Then there exists $p \in \R^A \setminus \{0\}$ and $k \in\R$ such that 
		\begin{enumerate}
			\item for all $q \in Q$, $q \cdot p \leq k$,
			\item for all $q' \in Q'$, $q \cdot p \geq k$. 
		\end{enumerate}
\end{theorem}



\section*{Second Fundamental Theorem of Welfare Economics}

\begin{theorem}
	Suppose that $\{I, J, (u^i, w^i, (s^{i,j}), Y^j )$ is a production economy where all $u^i$ are continuous, locally nonsatiated, and quasiconcave, and each set $Y^j$ is convex and satisfies free disposal.  Let $(\hat{x}, \hat{y})$ be a Pareto efficient allocation such that for all $i$, $x^i \gg0$. Then there exists prices $p$ and nominal incomes $(m^1, \hdots, m^I)$ such that 
	\begin{enumerate}
		\item $\sumi m^i = p \cdot \sum w^i + p  \cdot \sumj \hat{y}^j$,
		\item for all $i$, $\hat{x}^i$ maximizes $u^i(x)$ subject to $p  \cdot x \leq m^i$,
		\item for all $j$, $\hat{y}^j$ maximizes $p \cdot y$ subject to $y \in Y^j$,
		\item $\sumi \hat{x}^i = \sumi w^i + \sumi \hat{y}^j$.
	\end{enumerate}
\end{theorem}

Point (d) is actually implied by Pareto efficiency, but whatever. 


\paragraph{The Setup.} Suppose $(\hat{x}, \hat{y})$ is a Pareto efficient allocation such that for all $i$, $x^i \gg0$. We'll be thinking of this allocation as one that a social planner wants to implement. For simplicity, we will assume that $I=2$ and $J=1$. 

Define the set 
	\[	\mathcal{U}^i  =\{ x^i \;|\; u^i(x^i) > u^i(\hat{x}^i)\} .\]
So $\mathcal{U}^i$ consists of all bundles that individual $i$ strictly prefers to the Pareto efficient bundle. Define 
	\[	\mathcal{U} = \mathcal{U}^1 + \mathcal{U}^2 = \{ x \;|\; \exists  x^1 \in \mathcal{U}^1 \text{ and } \exists x^2 \in \mathcal{U}^2 \text{ satisfying } x^1 + x^2 = x.\} \]
	So $\mathcal{U}$ is the set of points that can be written as a sum of one point from $\mathcal{U}^1$ and one point from $\mathcal{U}^2$. 

Also define 
	\[	\mathcal{F} = \sumi w^i + Y = \{ x \; | \; \exists y \in Y \text{ satisfying } y + \sumi w^i = x \}.	\]
This is the set of all feasible points for the planner; it represents all possible combinations of commodities the economy could potentially have. 


\begin{claim} \emph{	$\mathcal{U}^i$ are convex.} This follows from quasiconcavity of utility functions.  For $x, \tilde{x} \in \mathcal{U}^i$, it follows for any $\lambda >0$ that 
	\[	\ u^i( \lambda x + [1 - \lambda] \tilde{x} ) \geq \min \{ u^i(x), u^i(\tilde{x})\} > u^i(\hat{x}^i) . \qedhere \]
\end{claim}

\begin{claim}
	\emph{$\mathcal{U}$ is convex}. It is the sum of convex sets. 
\end{claim}

\begin{claim}
	\emph{$\mathcal{F}$ is convex}. This follows because $Y$ is assumed convex. 
\end{claim}

\begin{claim}
	$\mathcal{U} \cap \mathcal{F} = \emptyset$. If $x \in \mathcal{U}$ and $x \in \mathcal{F}$, then we can find some $x^1 \in \mathcal{U}^1$ and $x^2$ such that $u^1(x^1) > u^1(\hat{x}^1)$ and $u^2(x^2) > u^2(\hat{x}^2)$ where $x^1 + x^2 =x$. Furthermore, we can find some $y \in Y$ such that $y + w^1 + w^2 = x$, i.e. is feasible. Because $x$ is both feasible and superior to $\hat{x}$, it follows that $\hat{x}$ cannot be Pareto efficient. By contradiction, the result follows. 
\end{claim}

\begin{claim}
	\emph{There exists some $p \in \R^L \setminus \{0\}$ and some $k \in \R$ such that}
	\begin{enumerate}
		\item \emph{for any $x \in \mathcal{U}$, $p \cdot x \geq k$,}
		\item \emph{for any $x' \in \mathcal{F}$, $p \cdot x' \leq k$.}
	\end{enumerate}
	This follows because $ \mathcal{U}$ and $ \mathcal{F}$ are convex and disjoint, and therefore we can apply the separating hyperplane theorem. 
\end{claim}

\begin{claim}
	\emph{$p \gg 0$.} This follows because $Y$ satisfies free disposal. Suppose, for instance, that $p_1 \leq 0$. We're essentially saying that a firm gets paid to absorb as many inputs as possible with no downside whatsoever. So a firm would choose $y_1 = -\infty$ and profits would blow up. More specifically, $p (w^1 + w^2 + y) \rightarrow \infty$, which is in $\mathcal{F}$. And thus, whatever $k$ happens to be in the above claim, there exists some $x' \in \mathcal{F}$ such that $x' > k$, which is a contradiction. 
\end{claim}

\begin{claim}
	Since $(\hat{x}, \hat{y})$ is feasible, we know that $\hat{x}^1 + \hat{y}^1  = w^1 + w^2 + \hat{y} \in \mathcal{F}$ and therefore $p \cdot  (w^1 + w^2 + \hat{y}) = p \cdot (\hat{x}^1 + \hat{x}^2) \leq k$.
\end{claim}

\begin{claim}
	Suppose that $u^1(x^1) \geq u^1(\hat{x}^1)$ and $u^2(x^2) \geq u^2(\hat{x}^2)$. By the local nonsatiation of preferences, we can find some bundle $x^i(n) \in \mathcal{U}^i$ such that $\norm{x^i(n) - x^i} \leq 1/n$ for any $n \in \N$. It follows that $p \cdot[ x^1(n)+x^2(n)] \geq k$ for all $n$. In the limit, we clearly have $x^i(n) \rightarrow x^i$. From continuity it follows that that $p \cdot[ x^1+x^2] \geq k$
	
	The bundles $\hat{x}^1$ and $\hat{x}^2$ satisfy the antecedent, so it follows that $p \cdot ( \hat{x}^1+\hat{x}^2 ) \geq k$. Combined with the previous claim, it follows that $p \cdot[ \hat{x}^1+\hat{x}^2] = k$.
\end{claim}


Okay, enough with the claims---now we can get to the main points in the theorem itself.
\begin{enumerate}
	\item Let $m^i = p \cdot \hat{x}^i$ for each $i$. Then
		\[\sumi m^i = \sumi p \cdot \hat{x}^i = p  \cdot  \sumi \hat{x}^i = p\cdot \left(\sumi w^i + \hat{y} \right).	\]
	\item Suppose $x^1$ satisfies $u^1(x^1) \geq u^1(\hat{x}^1)$. From claim 8, $p \cdot (x^1 + \hat{x}^2) \geq k$. We also know from claim 8 that $p \cdot ( \hat{x}^1 + \hat{x}^2) = k$. It follows that $p \cdot x^1 \geq p \cdot \hat{x}^1 = m^1$. So any bundle that gives at least utility $u( \hat{x}^1)$ is at least as expensive as $\hat{x}$, making $\hat{x}^1$ the expenditure minimizer. Now appeal to duality---because preferences are continuous and locally nonsatiated and the price vector is strictly positive, it must be the case that $\hat{x}^1$ maximizes utility subject to $p \cdot x \leq m^1$. The optimality of bundle $\hat{x}^2$ follows similarly. 
	\item Fix $y \in Y$. We know that $p \cdot (w^1 + w^2 + y) \leq k$. We also know that $p \cdot (\hat{x}^1 + \hat{x}^2 )= p \cdot (w^1 + w^2 + \hat{y})= k$. It follows that $y \leq \hat{y}$. So part (c) of the claim has been proved, namely, that $\hat{y}$ is the profit maximizer. 
\end{enumerate}






\end{document}
	